\documentclass[a4paper,11pt]{book}
%\documentclass[a4paper,twoside,11pt,titlepage]{book}

\usepackage{listings}
\usepackage[utf8]{inputenc}
\usepackage[spanish]{babel}
\usepackage{fancyhdr}
\usepackage{graphicx}
\usepackage{afterpage}
\usepackage{float}
%\usepackage[pdfborder={000}]{hyperref} %referencia
\usepackage[colorlinks=true, allcolors=black]{hyperref}
\usepackage{longtable}
\usepackage{pdfpages}
\usepackage{url}
\usepackage{colortbl,longtable}
\usepackage[stable]{footmisc}
\usepackage{appendix}
\usepackage{index}
\usepackage{amssymb}
\usepackage{booktabs}
%\usepackage[style=list, number=none]{glossary} 
%\usepackage{titlesec}
%\usepackage{pailatino}
\usepackage[chapter]{algorithm}
\usepackage[chapter]{algorithmic}
%\usepackage{doxygen/doxygen}
%\usepackage[style=long, cols=2,border=plain,toc=true,number=none]{glossary}
\usepackage{glossaries}
\RequirePackage{verbatim}
%\RequirePackage[Glenn]{fncychap}

\decimalpoint
\usepackage{dcolumn}
\newcolumntype{.}{D{.}{\esperiod}{-1}}
\makeatletter
\addto\shorthandsspanish{\let\esperiod\es@period@code}
\makeatother


% ********************************************************************
% Re-usable information
% ********************************************************************
\newcommand{\myTitle}{Diagnóstico de tumores cerebrales a partir de IRM mediante aprendizaje profundo \xspace}
\newcommand{\mySubTitle}{Diseño e implementación de arquitecturas neuronales para la clasificación y la segmentación \xspace}
\newcommand{\myTitleENGLISH}{Brain tumor diagnosis from MRI images using Deep Learning \xspace}
\newcommand{\mySubTitleENGLISH}{Design and neuronal architecture implementation for classification and segmentation \xspace}
\newcommand{\myDegree}{Grado en Ingeniería Informática \xspace}
\newcommand{\myName}{Jaime Castillo Uclés\xspace}
\newcommand{\myProf}{Rosa María Rodríguez Sánchez\xspace}
\newcommand{\myFaculty}{Escuela Técnica Superior de Ingenierías Informática y de
	Telecomunicación\xspace}
\newcommand{\myFacultyShort}{E.T.S. de Ingenierías Informática y de
	Telecomunicación\xspace}
\newcommand{\myDepartment}{Ciencias de la Computación e Inteligencia Artificial \xspace}
\newcommand{\myUni}{\protect{Universidad de Granada}\xspace}
\newcommand{\myLocation}{Granada\xspace}
\newcommand{\myTime}{\today\xspace}
\newcommand{\myVersion}{Version 0.1\xspace}

\hypersetup{
	pdfauthor = {\myName jaimecastillo@correo.ugr.es},
	pdftitle = {\myTitle},
	pdfsubject = {},
	pdfkeywords = {palabra\_clave1, palabra\_clave2, palabra\_clave3, ...},
	pdfcreator = {LaTeX con el paquete ....},
	pdfproducer = {pdflatex}
}

%\hyphenation{}

%\makeindex
\makeglossaries

% Definición de comandos que me son tiles:
%\renewcommand{\indexname}{Índice alfabético}
\renewcommand{\glossaryname}{Glosario}

\pagestyle{fancy}
\fancyhf{}
\fancyhead[LO]{\leftmark}
\fancyhead[RE]{\rightmark}
\fancyhead[RO,LE]{\textbf{\thepage}}
\renewcommand{\chaptermark}[1]{\markboth{\textbf{#1}}{}}
\renewcommand{\sectionmark}[1]{\markright{\textbf{#1}}}

\setlength{\headheight}{1.5\headheight}

\newcommand{\HRule}{\rule{\linewidth}{0.5mm}}
%Definimos los tipos teorema, ejemplo y definición podremos usar estos tipos
%simplemente poniendo \begin{teorema} \end{teorema} ...
\newtheorem{teorema}{Teorema}[chapter]
\newtheorem{ejemplo}{Ejemplo}[chapter]
\newtheorem{definicion}{Definición}[chapter]

\definecolor{gray97}{gray}{.97}
\definecolor{gray75}{gray}{.75}
\definecolor{gray45}{gray}{.45}
\definecolor{gray30}{gray}{.94}

\lstset{ frame=Ltb,
     framerule=0.5pt,
     aboveskip=0.5cm,
     framextopmargin=3pt,
     framexbottommargin=3pt,
     framexleftmargin=0.1cm,
     framesep=0pt,
     rulesep=.4pt,
     backgroundcolor=\color{gray97},
     rulesepcolor=\color{black},
     %
     stringstyle=\ttfamily,
     showstringspaces = false,
     basicstyle=\scriptsize\ttfamily,
     commentstyle=\color{gray45},
     keywordstyle=\bfseries,
     %
     numbers=left,
     numbersep=6pt,
     numberstyle=\tiny,
     numberfirstline = false,
     breaklines=true,
   }
 
% minimizar fragmentado de listados
\lstnewenvironment{listing}[1][]
   {\lstset{#1}\pagebreak[0]}{\pagebreak[0]}

\lstdefinestyle{CodigoC}
   {
	basicstyle=\scriptsize,
	frame=single,
	language=C,
	numbers=left
   }
\lstdefinestyle{CodigoC++}
   {
	basicstyle=\small,
	frame=single,
	backgroundcolor=\color{gray30},
	language=C++,
	numbers=left
   }

 
\lstdefinestyle{Consola}
   {basicstyle=\scriptsize\bf\ttfamily,
    backgroundcolor=\color{gray30},
    frame=single,
    numbers=none
   }


\newcommand{\bigrule}{\titlerule[0.5mm]}


%Para conseguir que en las páginas en blanco no ponga cabeceras
\makeatletter
\def\clearpage{
  \ifvmode
    \ifnum \@dbltopnum =\m@ne
      \ifdim \pagetotal <\topskip
        \hbox{}
      \fi
    \fi
  \fi
  \newpage
  \thispagestyle{empty}
  \write\m@ne{}
  \vbox{}
  \penalty -\@Mi
}
\makeatother

\begin{document}
	
\input{portada/portada}
\begin{titlepage}
 
 
\setlength{\centeroffset}{-0.5\oddsidemargin}
\addtolength{\centeroffset}{0.5\evensidemargin}
\thispagestyle{empty}

\noindent\hspace*{\centeroffset}\begin{minipage}{\textwidth}

\centering
%\includegraphics[width=0.9\textwidth]{imagenes/logo_ugr.jpg}\\[1.4cm]

%\textsc{ \Large PROYECTO FIN DE CARRERA\\[0.2cm]}
%\textsc{ INGENIERÍA EN INFORMÁTICA}\\[1cm]
% Upper part of the page
% 

 \vspace{3.3cm}

%si el proyecto tiene logo poner aquí
\includegraphics[width=0.35\textwidth]{imagenes/icono.png}
 \vspace{0.5cm}

% Title

{\Huge\bfseries BraTS UGR\\
}
\noindent\rule[-1ex]{\textwidth}{3pt}\\[3.5ex]
{\large\bfseries Una interfaz de uso médico para la clasificación y segmentación de tumores cerebrales \\[4cm]}
\end{minipage}

\vspace{2.5cm}
\noindent\hspace*{\centeroffset}\begin{minipage}{\textwidth}
\centering

\end{minipage}
%\addtolength{\textwidth}{\centeroffset}
\vspace{\stretch{2}}

 
\end{titlepage}



\chapter*{}

%\cleardoublepage
%\thispagestyle{empty}

\begin{center}
{\large\bfseries \myTitle: \mySubTitle}\\
\end{center}
\begin{center}
\myName\\
\end{center}

%\vspace{0.7cm}
\noindent{\textbf{Palabras clave}: palabra\_clave1, palabra\_clave2, palabra\_clave3, ......}\\

\vspace{0.7cm}
\noindent{\textbf{Resumen}}\\

Poner aquí el resumen.

\cleardoublepage
\thispagestyle{empty}

\begin{center}
{\large\bfseries \mySubTitleENGLISH: \mySubTitleENGLISH}\\
\end{center}
\begin{center}
\myName\\
\end{center}

%\vspace{0.7cm}
\noindent{\textbf{Keywords}: Keyword1, Keyword2, Keyword3, ....}\\

\vspace{0.7cm}
\noindent{\textbf{Abstract}}\\

Write here the abstract in English.

\chapter*{}
\thispagestyle{empty}


\noindent\rule[-1ex]{\textwidth}{2pt}\\[4.5ex]

Yo, \textbf{\myName}, alumno de la titulación \myDegree de la \textbf{Escuela Técnica Superior
de Ingenierías Informática y de Telecomunicación de la Universidad de Granada}, con DNI 45924736S, autorizo la
ubicación de la siguiente copia de mi Trabajo Fin de Grado en la biblioteca del centro para que pueda ser
consultada por las personas que lo deseen.

\vspace{6cm}

\noindent Fdo: \myName

\vspace{2cm}

\begin{flushright}
Granada a X de Junio de 2024 .
\end{flushright}


\chapter*{}
\thispagestyle{empty}

\noindent\rule[-1ex]{\textwidth}{2pt}\\[4.5ex]

Dña. \textbf{\myProf}, Profesor del Área de \myDepartment del Departamento de \myDepartment de la Universidad de Granada.


\vspace{0.5cm}

\textbf{Informan:}

\vspace{0.5cm}

Que el presente trabajo, titulado \textit{\textbf{\myTitle, \mySubTitle}},
ha sido realizado bajo su supervisión por \textbf{\myName}, y autorizamos la defensa de dicho trabajo ante el tribunal
que corresponda.

\vspace{0.5cm}

Y para que conste, expiden y firman el presente informe en Granada a X de mes de 2024 .

\vspace{1cm}

\textbf{La directora:}

\vspace{5cm}

\noindent \textbf{}

\chapter*{Agradecimientos}
\thispagestyle{empty}

       \vspace{1cm}


Poner aquí agradecimientos...
\frontmatter
\tableofcontents
\listoffigures
\listoftables

\mainmatter
\setlength{\parskip}{5pt}

\chapter{Introducción}

Los tumores cerebrales son una de las formas más letales de cáncer. Específicamente, los glioblastomas y sus variantes difusas son los más comunes y agresivos tipos de tumor del sistema nervioso central en adultos. Su alta heterogeneidad en apariencia, forma e histología los convierte en una de las patologías más difíciles de diagnosticar y un reto para el campo de la imagen médica.

Sin duda la aplicación de técnicas de Visión por Computador es una de las máximas para la investigación en imagen médica en la actualidad. Sólo considerando su aplicación en el diagnóstico de enfermedades, desde 2008 el número de publicaciones promedio realizadas por año se ha incrementado notablemente tanto que actualmente es diez veces mayor que en sus inicios.


\section{Objetivos}

**Hablar sobre porqué al dejar la tarea de segmentar a una máquina esta puede ser más precisa que un humano**

Con este trabajo se pretende perseguir la creación de una arquitectura que mejore el estado del arte actual para equipar a un programa al servicio de personal médico para la ayuda en la evaluación del diagnóstico de un posible paciente de tumor cerebral.

A continuación, detallaremos de una forma más profunda estos objetivos.

El cerebro no tiene terminaciones nerviosas. Los pacientes no sienten dolor al causa de un tumor cerebral por sí mismo. Generalmente, acaban buscando ayuda médica por la aparición de otros indicios relacionados difíciles de distinguir de otras patologías agudas y de mucho menor transcendencia (visión borrosa, pérdida del control)

Los glioblastomas y sus variantes tienen una media de supervivencia de 15 meses tras su diagnóstico.

** Hablar sobre por qué el tratami
En general, los tumores cerebrales son difíciles de tratar y son resistentes a terapias convencionales usadas en otros tipos de cánceres como la quimioterapia debido a los desafíos que presenta el cerebro para tolerar ciertos químicos, transportar medicamentos dentro de él y la alta importancia que tiene en este órgano la optimización del uso de tratamientos que puedan ser invasivos. En otras palabras, el uso de tratamientos basados en la extirpación o en la medicación pueden ser arriesgados. Por tanto, el tratamiento más común de estos está basado en la radioterapia.


\section{Metodología}

\chapter{Estado del arte}

En este capítulo estudiaremos y analizaremos los diferentes enfoques dados históricamente para nuestras tareas, \textbf{clasificación de tumores cerebrales} y \textbf{la segmentación de tumores cerebrales}. Abordando desde el inicio del estudio del problema pasando por la explosión de métodos basados en Aprendizaje profundo con la constitución de \textbf{BraTS} hasta nuestros días. Se pondrá especial énfasis a las soluciones actuales comparándolas desde sus diferencias en metodología y perspectiva.

Por un lado, la clasificación entre los dos tipos de tumores no es tan relevante a la hora de diseñar un sistema de ayuda a la toma de decisión ya que clínicamente sí existe una característica diferencial entre ambos, su \textbf{localización}. Los meningiomas aparecen entre el cráneo y el cerebro no internamente en el cerebro como los glioblastomas. Esto hace que un médico pueda distinguirlos sin requerir una gran asistencia para la mayoría de los casos. No obstante, la clasificación de por sí ayuda a la toma de decisiones en el tratamiento pero no es el elemento crítico para la supervivencia del paciente que depende de la eliminación del tumor donde su segmentación toma un papel crucial. Por esto, veremos como el estado del arte del problema de segmentación es mucho mayor que el del problema de clasificación.

Por otro lado, la tarea de predicción de la evolución del tumor debido a la baja densidad de instancias temporales de los datos existentes hacen que este problema no sea tratado en trabajos. No hemos encontrado literatura al respecto. 

Los grandes esfuerzos se han realizado en entorno a la segmentación, ya que otras tareas que conforman su diagnóstico se verían arrastradas.

\section{Revisión histórica de clasificación binaria de tumores}

\section{Revisión histórica de segmentación}

Diferentes dificultades han sido las que a pesar de años de desarrollo aún encontrar un algoritmo para la segmentación de tumores cerebrales sea algo mejorable. 

\begin{enumerate}
	\item \textbf{Incertidumbre en la localización} : Como vimos no existe una zona concreta en general para la aparición de los tumores cerebrales. A excepción, de los meningiomas localizados en zonas superficiales del cerebro y aún siendo una región muy amplia, incluso ya desarrollado un tumor pueden aparecer otros localizados en regiones muy distintas de la original. 
	\item \textbf{Incertidumbre en la morfología} : A diferencia de otras patologías, cada tumor cerebral presenta un tamaño y forma completamente distintas y donde en principio no se puede apreciar un patrón distintivo. Esto hace que sea muy complicado y generalmente aporte malos resultados, la construcción de sistemas basados reglas u otras aproximaciones que no incluyen una componente de aprendizaje.
	
	\item \textbf{Bajo contraste} : Una buena resolución y contraste son características muy importantes para entender la información de una imagen. Las imágenes IRM producidas en una resonancia debido a proyecciones de imagen y procesos de tomografía usualmente ofrecen una baja resolución y contraste haciendo más difícil la definición de bordes entre diferentes tejidos de la imagen. Una segmentación precisa es difícil de conseguir.
	
	\item \textbf{Sesgo en las etiquetas}. Existen indicios para pensar que las etiquetas proporcionadas pueden presentar ruido. El proceso de segmentado por parte del personal médico 
	depende de su experiencia profesional lo cual puede llevar a cometer errores. Por ejemplo, se han presentado eventualmente discrepancias entre distintos anotadores: algunos tienden a conectar todas las pequeñas regiones de un tejido mientras que otros las segmentan de forma más precisa y separada. 
	
	\item \textbf{Desbalanceo en el tejido} : Dentro de la segmentación entre los diferentes tipos de tejidos, usualmente existe un tejido NCR que es usualmente más pequeño que los otros dos. 
	Esto podría afectar en el proceso de aprendizaje hacia una pobre generalización de este tipo de tejido. 
	
	\item \textbf{Desbalanceo entre pacientes} : En el conjunto de datos tenemos muchos pacientes de norteamerica y de ascendencia blanca, pero pocos de otros origenes como el africano. Además, de tener un sesgo claro de edad ya que existen pocos casos en niños. Esta falta de datos puede impedir que exista una buena generalización para estos casos más aislados.
	
\end{enumerate}


	A continuación, se presenta una revisión histórica sobre la segmentación de tumores cerebrales hasta 2021 apoyada en \cite{liu2023deep}. Se presenta una línea del tiempo con los principales trabajos de estudio.
	
	\begin{figure}[H]
		\centering
		\includegraphics[width=1.0\linewidth]{imagenes/evolution_stateofart.png}
		\caption{Evolución histórica del estado del arte hasta 2021.}
	\end{figure}
	
	En la década de 1990 investigadores como \cite{zhu1997computerized} fueron pioneros al utilizar una red Hopfield con un modelo de contornos activos para extraer los bordes del tumor. Sin embargo, incluso el entrenamiento de una pequeña red como esta era algo computacionalmente costoso por las limitaciones de la época.  Desde 1990 hasta 2012, los métodos que iban surgiendo para la segmentación de tumores cerebrales estaban basados en métodos clásicos de aprendizaje con características extraídas a mano, sistemas expertos que se apoyaban en los histogramas de la imagen, plantillas para la segmentación y modelos gráficos. 
	
	A pesar de ser un gran paso inicial, tenían grandes deficiencias. Por ejemplo, la mayoría de ellos sólo se centraba en la segmentación de todo el tumor lo cual lleva a un modelo poco útil. Por otro lado, en los modelos basados en características extraídas se hacía muy tedioso poder usarlos eficazmente ya que este paso de extracción dependía de conocimiento previo experto que en ningún momento se pudo llegar a representar en un modelo. En último lugar, los mismos problemas que compartimos hoy en día sobre el desbalanceo y la incertidumbre del problema eran mucho más agresivos. 
	
	Tras 2012 con la revolución del Deep Learning, se introducen nuevas tecnologías (Redes neuronales convolucionales y U-net) que mejorarán los resultados obtenidos hasta el momento. 
	Se empezarán a construir arquitecturas encoder-decorder convolucionales para conseguir pipelines completos para la segmentación. El aprendizaje profundo toma el problema de lleno proclamándose el enfoque que define el estado del arte.
	
	Podemos clasificar las soluciones basadas en aprendizaje profundo aportadas en tres categorías según el principal problema para que el están pesadas. Sin embargo, como veremos en las soluciones más actuales lo ideal es tratar con los tres problemas.
	
	\subsection{Métodos que se enfocan en la arquitectura}
		
		 Para poder obtener redes que automáticamente extraen características discriminativas a altas dimensiones es necesario un efectivo diseño de módulos y arquitecturas. Por un lado, se pretende que la arquitectura sea capaz de aprender las características distintivas de los tejidos y a localizar regiones de interés por medio de añadir profundidad a la red, a través de mecanismos de atención o la fusión de características entre las resonancias. Por otro lado, se pretende minimizar la cantidad de parámetros entrenables de la red o conseguir un entrenamiento más rápido.
		 
		\subsubsection{Diseño de bloques especializados}
			
			Los primeros trabajos que tenían este objetivo comenzaron por basarse en arquitecturas bien conocidas como AlexNet o VGGNet a través del uso de una única imagen de la resonancia completa como entrada de la red.
			
			Para la mejora de resultados, se optó por introducir todas la secuencia de imágenes de una resonancia como entrada de la red y añadir más capas convolucionales. Con ello, teníamos redes más profundas pero que pronto empezaban a sufrir los problemas de la explosión y desvanecimiento del gradiente durante el proceso de entrenamiento. Para ayudar a lidiar con estos problemas, se introdujo a las redes, \textbf{conexiones residuales} \cite{chang2016fully}. Conectando la entrada de la red con su salida, convergiendo más rápido y con mejores resultados. 
			
			Este proceso de aumento de profundidad con conexiones residuales no sería definitivo porque también conlleva el sacrificio de resolución espacial. Se reemplazaría en trabajos siguientes, el uso de la convolución simple por convoluciones dilatadas. El \textbf{uso de convoluciones dilatadas} traería el aumento del espacio receptivo (ya que se aplica una convolución a un espacio mayor de la imagen) sin necesidad de introducir parámetros a la red. La convolución dilatada se vería especialmente útil por ejemplo en la segmentación de áreas grandes como suele ocupar el tejido ED (edema tumoral). 
			
			Respecto conseguir una buena eficiencia en tiempo de entrenamiento es conocido aplicar un reordenamiento en memoria de las imágenes de la resonancia similares (p. ej. el mismo slice en las 4 pruebas) de forma que se reduzcan la comunicación entrada-salida con GPU. Adicionalmente, autores como \cite{brugger2019partially} utilizan \textbf{conexiones reversibles}  en la red de forma que durante el proceso de backpropagation (backward pass) no se necesite memoria adicional para guardar las activaciones intermedias. Por último, para ahorrar en eficiencia se sustituye la convolución standard por la combinación de \textbf{convoluciones separables}.
			
			\subsubsection{Diseño de arquitecturas efectivas}
			
			La mayoría de los trabajos de recorrido histórico se encasillan en alguno de los siguientes dos enfoques de arquitectura: \textbf{redes neuronales convolucionales} para extraer características de la imagen y clasificar los patches o píxeles de la imagen según las etiquetas de los tejidos posibles o \textbf{redes encoder-decoder} en las cuales se puede definir un pipeline completo convolucional sin la necesidad de la agregación de capas totalmente conectadas.
			
			\begin{enumerate}
					
				\item \textbf{Redes neuronales convolucionales de una/múltiples trayectorias}
					
				A diferencia de una red convolucional de una única trayectoria, las redes de trayectoria múltiples tienen la capacidad de extraer diversas características a diferentes escalas. Estas características se combinan para su posterior procesamiento, usualmente en capas totalmente conectadas, permitiendo a las redes aprender tanto características globales como locales. 
				
				\begin{figure}[H]
					\centering
					\includegraphics[width=0.27\linewidth]{imagenes/comparisonsinglemultipleCNN.png}
					\caption{Comparación entre arquitecturas de una y múltiples trayectorias. Imagen de \cite{liu2023deep}}
				\end{figure}
				
				Por ejemplo, \cite{havaei2017brain} desarrollaron una estructura de dos vías que integra información tanto local como global del tumor, utilizando núcleos de convolución de diferentes tamaños.
				
				\begin{figure}[H]
					\centering
					\includegraphics[width=0.75\linewidth]{imagenes/havaei2017architecture.png}
					\caption{Arquitectura de dos vías de \cite{havaei2017brain}}
				\end{figure}

				Otros enfoques, como el de \cite{kamnitsas2017efficient}, optan por aprender información global y local desde la entrada misma, utilizando redes de doble vía, patches de diferentes tamaños y pequeños núcleos de convolución. 
				
				Este tipo de arquitecturas fueron una de las primeras aproximaciones que empezaban adaptarse con éxito a las complejidades de la segmentación de tumores cerebrales. Sin embargo, veremos como la dificultad de un buen ajuste en el diseño de estas arquitecturas todavía seguía siendo un problema.
					
				\item \textbf{Arquitecturas Encoder-Decoder}
				
				Las redes de una/múltiples trayectorias toman como input un patch de una cierta región de la imagen y dan como output la clasificación del tejido que existe en ese patch. Este enfoque hace que obtener una buena arquitectura que haga la transformación de los patches a información categórica sea complicado por varios motivos: 
				\begin{enumerate}
					\item Existe una gran \textbf{dependencia} entre el tamaño y calidad de los patches, y los resultados que ofrecería la arquitectura.
					
					\item Toda la transformación de características visuales (aunque, reducidas) a información categórica estaría concentrada en las capas totalmente conectadas. Las capas totalmente conectadas de un tamaño razonables para una capacidad de memoria usualmente utilizada \textbf{no puede totalmente representar un espacio de características tan grande}.
					
					\item Si necesitamos tener distintas redes separadas, el proceso de ajuste de cada una de ellas es independiente. Esto lo podemos interpretar como un coste añadido en términos de \textbf{eficiencia}.
					
				\end{enumerate}
				
				Para superar estos problemas en los siguientes trabajos se empieza a utilizar \textbf{FCN Redes neuronales totalmente convolucionales} y \textbf{U-net} basadas en arquitecturas encoder-decorder, de forma que se establece un pipeline completo desde la imagen a la segmentación.
				
				Una de los tipos más importantes de FCN para este problema es U-net. U-net consiste en la creación de conexiones entre el encoder y el decoder. Permitiendo una vinculación directa en el proceso de reducción y ampliación de dimensionalidad. Estas conexiones reciben el nombre de \textbf{Skip Connections} y pueden ayudar a las capas del decoder a recuperar detalles visuales aprendidos en el encoder, llevando a una segmentación más precisa.
				
				\cite{isensee2018brain} utilizan una U-Net dándole aún más énfasis a la tarea de una segmentación utilizando una función de pérdida basada en la similaridad Dice.
				
				Similar a las skip connections antes mencionado, el uso de conexiones residuales  y skip connections permiten el paso de características de alto y bajo nivel para una mejor segmentación final.
				
				\begin{figure}[H]
					\centering
					\includegraphics[width=0.85\linewidth]{imagenes/encoder-decoderIMG.drawio.png}
					\caption{Comparación de distintas arquitecturas encoder-decoder}
				\end{figure}
				
			\end{enumerate}
			
			
		\subsection{Métodos que tratan el desbalanceo}
		
		Como anunciábamos anteriormente el alto desbalanceo en la dimensión de los diferentes tejidos del tumor (usualmente, ocupando NCR regiones pequeñas, y ED grandes) puede tener un impacto negativo en el proceso de entrenamiento. Motivados por métodos como los sistemas multi-expertos, se empezó a construir métodos específicos para este problema.
		
		Podemos diferenciar en:
		\begin{enumerate}
			\item \textbf{Diseños sobre la arquitectura}: Redes en cascada, ensamblado de modelos y arquitecturas multi-tarea.
			\item \textbf{Mejorar el entrenamiento}: Funciones de pérdida especializadas.
		\end{enumerate}
		
			\subsubsection{Redes en cascada}
			
			Una red en cascada es un conjunto de redes más pequeñas ordenadas en las cuales el output de la red anterior sirve como una input a la siguiente, formando una  <<cascada de redes>>. De esta forma, podemos tener redes especializadas en distintos niveles. 
			
			Las primeras redes de la cascada especializadas a características de más alto nivel y las siguientes de más bajo.
			
			Por ejemplo, en \cite{wang2018automatic} se utilizan tres redes especializadas para los tres regiones de tejidos definidas por BraTS. Empezando por la región más grande hasta la más pequeña. 
			
			Su primera red WNet segmenta a Whole Tumor, toda la lesión. La siguiente TNet segmenta al núcleo del tumor. Finalmente, Enet a la parte activa del tumor.
			
			
			\begin{figure}[!h]
				\centering
				\includegraphics[width=0.5\linewidth]{imagenes/cascadestructure.png}
				\caption{Estructura de método en cascada de \cite{wang2018automatic}}
			\end{figure}
			
			La ventaja de este modelo es evitar la interferencia de las clases desbalanceadas, ya que cada red trata su clase como un problema de segmentación binaria. 
			
			Sin embargo, hace que la redes dependientes de otras dependan también de sus resultados. Si la primera red obtiene malos resultados, todas las siguientes redes se verán afectadas por ella.
					
			\subsubsection{Ensamblado de modelos}
			
			Una de las consecuencias que tiene el uso de una sola red es que está altamente influenciada por la elección de su hiperparámetros. Con el objetivo de obtener un más robusto y general modelo para la segmentación se puede combinar la salida de múltiples redes, ensamblarlas.
			
			El ensamblado de modelos aumentaría el espacio de hipótesis del modelo final evitando, la caída en óptimos locales debido a el desbalanceo de datos.
			
			EMMA  de \cite{kamnitsas2018ensembles} es uno de los primeros modelos para segmentación de tumores que es un ensamblado de varias redes. EMMA utiliza tres modelos: DeepMedic, una red FCN y una U-net para dar el output de los tres con una mayor confianza.
			
			\cite{jiang2020two} ganadores de BraTS2019 adoptaron una estrategia de ensamblado con $12$ modelos obteniendo entorno $0.6 - 1 \%$ mejores resultados que el mejor único modelo.
			
			\subsubsection{Arquitecturas multi-tarea}
			
			Todo lo descrito en esta revisión histórica gira entorno a la segmentación de tumores. Sin embargo, la desventaja que puede tener enfocarnos en esta sola tarea es que quizá los modelos específicos para segmentación ignoran información útil en las imágenes para otras tareas, que indirectamente pueda ayudar a obtener una mejor generalización en la segmentación de tumores. 
			
			Por un lado, esta idea radica en la suposición de que los modelos que aprenden más tareas están aumentando su aprendizaje en el dominio del problema y esto debería ser beneficioso para todas las tareas. Por otro lado, de una forma más justificada, sabemos que nos enfrentamos a cierto ruido que desconocemos en los datos y etiquetas por tanto si entrenamos para múltiples tareas en conjunto el modelo aprende representaciones más generales reduciendo el riesgo de sobreajuste. Añadir tareas a la arquitectura y aprenderlas en conjunto podría tener \textbf{un efecto regularizador}.
			
			Un claro ejemplo de esto es \cite{myronenko20193d} que usa como tarea complementaria la reconstrucción de la resonancia de entrada mediante un autoencoder. Teniendo un efecto regularizador sobre los parámetros compartidos del encoder que a diferencia de regularizaciones L1 o L2 que explícitamente añaden una penalización para evitar el sobreajuste, la tarea nueva añade una penalización en la dirección en la que ambas tareas son optimizadas reduciendo el espacio de búsqueda de los parámetros entrenables de la red.
			
			\begin{figure}[H]
				\centering
				\includegraphics[width=1.0\linewidth]{imagenes/myroenko2019.png}
				\caption{Arquitectura del autoencoder regularizador de \cite{myronenko20193d}}
			\end{figure}
			
				
			\subsubsection{Funciones de pérdida especializadas}
			
			De forma más detallada, el problema del desbalanceo entre los diferentes tejidos se manifiesta durante el proceso de entrenamiento, en un gradiente excesivamente influenciado por los tejidos mayoritarios. Por ello, atacando directamente al problema multitud de trabajos proponen funciones de pérdida especializadas.
			
			Funciones de pérdida estándar en este problema incluyen categorical cross-entropy, cross-entropy y dice loss $D_{L}$.
			
			Una de las aproximaciones es el uso de utilizar una función de pérdida balanceada. Por ejemplo, añadir una penalización en función de la presencia del tejido segmentado para mitigar su escasa presencia respecto el total.
			
			Otro enfoque se basa en la combinación de diferentes funciones de pérdida en una nueva. Por ejemplo, una nueva función de pérdida de cross-entropy a nivel de píxel y dice loss podría ser su media.
			
			En general, funciones de pérdida que eviten el desbalanceo y mejoren el nivel de atención de las arquitecturas es beneficioso a todo tipo de problemas. Por ello, a diferencia de seguir funciones clásicas como cross-entropy, \cite{lin2017focal} proponen una nueva función llamada \textbf{Focal Loss} que será vista en años recientes en combinación con Dice Loss para diversos problemas de segmentación.
			
			
		\subsection{Métodos que tratan la información multi-modal}
		
		Las imágenes asociadas a una resonancia contienen diferentes tipos de imagen según las características de la frecuencia y contraste suministrado al paciente en su toma. Esta forma de proceder en la toma de resonancias es debido a las limitaciones de las imágenes IRM de poder representar y al menos para el ojo humano visualizar todos los tejidos importantes en el diagnóstico. Por ello, surge como idea clave tener métodos que tengan los objetivos de poder fusionar, relacionar y incluso distinguir en importancia las diferentes modalidades de imagen.
		
		Otras arquitecturas basadas en autoencoders como \cite{myronenko20193d} únicamente fusionan las cuatro modalidades como los canales de una imagen para un mismo slice concatenando las cuatro pruebas en la misma entrada, obteniendo entradas de dimensiones $ H \times W \times 4 $ en caso de 2D y $ H \times W \times D \times 4$ en caso de 3D.
		
		Sin embargo, usar concatenación o adición como método de fusión de los cuatro métodos no permitiría a la red de una forma directa aprender semánticamente la relación entre ellas. Por ello, en trabajos recientes se han adoptado mecanismos de atención aplicados a hacer aprender a la red de forma más robusta las diferentes modalidades e información espacial.
		
		\cite{zhou2021latent} proponen también una arquitectura encoder-decoder con la particularidad de crear un encoder y decoder especifico para cada una de las cuatro posibles representaciones, teniendo un espacio latente donde se fusiona la información de salida de los cuatro encoder dando un tratamiento especial a la fusión de las diferentes pruebas.
		
		A continuación, podemos ver la arquitectura especifica usada.
		
		\begin{figure}[H]
			\centering
			\includegraphics[width=0.75\linewidth]{imagenes/latentcorrelationrepresentation.png}
			\caption{Arquitectura de \cite{zhou2021latent}}
		\end{figure}
		
		Por un lado, transforma las representaciones individuales a representaciones correlacionadas. A través de lo que denominan \textbf{correlation model}.
		
		\begin{figure}[H]
			\centering
			\includegraphics[width=0.75\linewidth]{imagenes/zhoufusionmodel.png}
			\caption{Modelo especializado en la correlación de las modalidades}
		\end{figure}
		
		El \textbf{correlation model} se compone de dos partes: módulo de estimación de parámetros (MPE) y un módulo de expresión de correlación lineal (LCE).
		
		El módulo de estimación de parámetros se compone de una de dos redes totalmente conectadas que vinculan cada representación salida de cada encoder con unos parámetros $ \Gamma_i = \{ \alpha_i , \beta_i , \gamma_i , \delta_i \}$
		
		El módulo de expresión de la correlación lineal (LCE) utiliza estos parámetros para obtener una versión correlacionada de cada representación individual aplicando: 
		$$ F_i (X_i | \theta_i ) = \alpha_i \odot \gamma_i f_j (X_j | \theta_j ) + \beta_i \odot f_k (X_k | \theta_k ) + \gamma_i \odot f_m (X_m | \theta_m ) + \delta_i , \quad (i \neq j \neq k \neq m)$$
		
		Tras ello, se fusiona las representaciones correlacionadas resultado. Permitiendo al modelo manejar de forma explícita la información multi-modal y dándole robustez ante pruebas faltantes. 
		
		\begin{figure}[H]
			\centering
			\includegraphics[width=1.1\linewidth]{imagenes/zhouarchitecture.png}
			\caption{Red \cite{zhou2021latent} de fusión de representaciones latentes}
		\end{figure}
		
		Si bien esta arquitectura da ligeramente peores resultados que \cite{myronenko20193d} define un paso más en el estado del arte al usar menos recursos computacionales.


\section{Enfoques actuales para la segmentación}

	Las soluciones más relavantes presentadas en la revisión histórica que se ha hecho anteriormente se basan en la aplicación de la convolución sobre las imágenes de resonancia magnética. En el diagnóstico de tumores cerebrales ha tenido largo recorrido el uso de redes neuronales convolucionales. 
	
	Con la inclusión de las arquitecturas transformadoras se planteó un nuevo modelo que podía traer ventajas significativas. No siendo la imagen médica y en concreto este problema una excepción.
	
	Con la adaptación de los transformers al campo de la visión, los Vision Transformers podría ser un modelo más unificador, paralelizable y que ofreciera mejores resultados que las redes convolucionales al romper con la localidad que supone el uso de convoluciones. 
	
	En las soluciones más recientes de la segmentación de tumores cerebrales se introduce el uso de Vision Transformers con estas expectativas.
	
	\subsection{Basados en Transformers}
	
	A continuación, se presentan las soluciones principales que hacen uso de una arquitectura basada en Transformers para la segmentación de tumores cerebrales.
	
		\subsubsection{}

\chapter{Metodología}

En este capítulo describimos en profundidad todos los pasos seguidos en los métodos empleados en el trabajo y su justificación. Posteriormente, se aplicarán en la experimentación.

\section{Análisis de los recursos disponibles}


\section{Preprocesado de Datos}

En este apartado se explicará el preprocesamiento que se ha aplicado a las resonancias magnéticas para convertirlas a entradas de los modelos. 


Partiendo de nuestro conjunto de datos que presentamos en la introducción obtenido de la competición BraTS en Synapse. Ya vemos como las resonancias presentan características favorables para ser una entrada a la red.

\begin{enumerate}
	\item \textbf{Dimensiones estandarizadas} : Todas las resonancias (adultos, niños, diferente tipo de tumor) presentan las mismas dimensiones.
	\item \textbf{Imágenes estandarizadas} : Dado todas las resonancias se han hecho con el mismo estándar de escáner, todas presentan el mismo rango para su visualización. 
	\item \textbf{No existen valores faltantes} : Observamos como el conjunto de datos es completo en su definición, todas las resonancias de cada paciente tienen las mismas cuatro pruebas.
\end{enumerate}

\subsection{Normalizado de las imágenes}

Las imágenes que componen las resonancias son mapas en escala de gris donde un píxel de la imagen puede tomar un valor de gris en el intervalo $[0, 256)$. Entre las imágenes de distintas resonancias se encuentra una misma distribución de valores de píxeles para representar la misma información. Sin embargo, el proceso de entrenamiento no deja de ser un proceso de optimización y puede que este rango sea aún demasiado grande.

Adicionalmente, para evitar posibles píxeles erróneos en la toma de las imágenes que podamos interpretar como outliers que tengan un impacto negativo en el entrenamiento y para hacer las imágenes más interpretables se aplica a las imágenes normalización Z-score o estandarización.

$$ X_{std}^{i}= \frac{x^{i}-mean}{std} $$



\subsection{Recortado de imagen}

Podría ser razonable reducir las dimensiones de las imágenes para hacer a nuestros datos menos pesados. Sin embargo, se opta por no hacerlo por seguridad y escalibilidad. BraTS fija esas dimensiones en base del estándar en una resonancia magnética, así para cualquier paciente se garantiza que la imagen de su cerebro se puede representar en una resonancia en unas condiciones de resolución iguales al resto de pacientes.

Si recortamos las imágenes de forma cuadrada al cerebro más grande de todas las resonancias, podríamos encontrarnos en inferencia con un cerebro mayor que no se podría representar en una imagen. Es necesario dejar cierto margen, optando por respetar el margen inicial que marcan los organizadores médicos de BraTS.


\chapter{Experimentación}

En este capítulo se recoge el desarrollo de los experimentos llevados a cabo en este trabajo y todos los resultados que se han obtenido empíricamente de estos.

\section{Bibliotecas y desarrollo de los experimentos}

Para el desarrollo de los experimentos usaremos las bibliotecas de aprendizaje profundo \textbf{PyTorch} para la construcción de la estructura del proyecto y \textbf{Fastai} biblioteca extensión de PyTorch para el proceso de entrenamiento de los modelos. Construiremos nuestras propias clases en PyTorch para la construcción de las arquitectura y para la creación de la instancia de trabajo de nuestro dataset. Tenemos las siguientes componentes.

\begin{enumerate}
	\item \textbf{Clase del Dataset: BraTS}. Clase que representa al conjunto de datos. Sus funciones principales son \texttt{len(self)} que devuelve la cantidad de datos en la clase y \texttt{getitem(self, index)} que devuelve un dato de la clase dado un índice. Esta clase hereda de la clase \textbf{Dataset} de PyTorch.
	La instancia de esta clase será la entrada a la clase \textbf{DataLoader} de PyTorch.
	
	\item \textbf{Clases de las arquitecturas: SegNet y BinaryNet}. Implementan las arquitecturas para clasificación y segmentación. Aparte del constructor que es donde se define la arquitectura tiene la función \texttt{forward(self, x)} que define como se infiere por la red. PyTorch solo necesita la definición de la inferencia para calcular automáticamente la función \texttt{backward} internamente la función para la aplicación de backpropagation (necesario en el entrenamiento). Usa las instancias de los módulos siguientes.
	
	\item \textbf{Clases de los módulos de subida y bajada: DownConv y UpConv}. Análogas a las clases de la arquitecturas pero reduciéndolo a una parte. Usa la instancia de la clase siguiente (ya que sólo definimos uno de estos bloques por módulo).
	
	\item \textbf{Clase de un bloque de convolución: ConvBlock}. Clase que define las capas de un bloque de en nuestra red. Dentro del constructor llama a las funciones de PyTorch de ReLU, convolución y batch normalization.
	
\end{enumerate}

A continuación, podemos ver en detalle la estructura de todos nuestros experimentos a través de un diagrama de clases.

\begin{figure}[H]
	\centering
	\includegraphics[width=0.75\linewidth]{imagenes/clases_pytorch.png}
	\caption{Diagrama de clases de PyTorch}
\end{figure}

\section{Construcción del codificador y representación latente}

Pasamos a comentar los experimentos llevados cabo en la tarea de la reconstrucción de imágenes.

\subsection{Arquitecturas con conexiones residuales: ResNet34}

Comenzamos ajustando \textbf{ResNet34} a las imágenes. Tras $7$ épocas sumando un total de $8$ horas $30$ minutos de entrenamiento para toda la red obtenemos estos resultados.
Podemos ver los resultados en forma tabular y en forma de gráfica. En la gráfica el eje $Y$ representa la pérdida y el eje $X$ representa la cantidad de imágenes vistas (aunque haya visto la misma imagen más de una vez en distintas épocas). 

\begin{table}[H]
	\centering
	\begin{tabular}{|ccccc|}
		\toprule
		epoch & train\_loss & valid\_loss & MAE & time \\ 
		\midrule
		0 & 0.098790 & 0.111301 & 0.111301 & 1:09:46 \\ 
		0 & 0.079261 & 0.081548 & 0.081548 & 1:13:33 \\
		2 & 0.071293 & 0.075421 & 0.075421 & 1:14:20 \\ 
		3 & 0.064748 & 0.067580 & 0.067580 & 1:14:41 \\ 
		4 & 0.059911 & 0.062933 & 0.062933 & 1:15:55 \\ 
		5 & 0.057501 & 0.059307 & 0.059307 & 1:15:36 \\ 
		6 & 0.055986 & 0.058405 & 0.058405 & 1:15:37 \\ 
		\bottomrule
	\end{tabular}
	\caption{Pérdida de entrenamiento y validación para la reconstrucción con ResNet34}
	\label{tabla:resultados2}
\end{table}

\begin{figure}[H]
	\centering
	\includegraphics[width=0.7\linewidth]{imagenes/curva_resnet34.png}
	\caption{Curva de aprendizaje con ResNet34}
\end{figure}

Observamos una rápida convergencia la inicio y un entrenamiento perfecto con ambos errores muy cercanos en todas las épocas $E_{val} \approx E_{train}$. Indicando que el proceso de entrenamiento está realizando correctamente y que la arquitectura es válida para aprender a reconstruir las imágenes. Finalmente, obtenemos un $E_{val} \approx 0.058$ lo cual indica que nuestra red reconstruye las imágenes con una pérdida del $5.8 \%$ de los detalles reales.

A continuación, observamos cómo la red reconstruye tres imágenes. En la siguiente salida vemos tres imágenes donde podemos apreciar la salida de la red como la imagen de la izquierda de cada pareja y la imagen real a la derecha.

\begin{figure}[H]
	\centering
	\includegraphics[width=0.4\linewidth]{imagenes/reconstruccion_resnet34.png}
	\caption{Reconstrucción de las imágenes con ResNet34}
\end{figure}

\subsection{Arquitecturas con filtros con distinto tamaño: Xception}

Dentro de las pruebas para elegir el mejor codificador seguimos con la prueba de bondad de \textbf{Xception}. Lo evaluamos en las mismas condiciones que \textbf{ResNet34}.

\begin{table}[H]
	\centering
	\begin{tabular}{|ccccc|}
		\toprule
		epoch & train\_loss & valid\_loss & MAE & time \\ 
		\midrule
		0 & 0.102650 & 0.105854 & 0.105854 & 1:24:38 \\ 
		1 & 0.083989 & 0.089979 & 0.089979 & 1:28:34 \\ 
		2 & 0.074420 & 0.085525 & 0.085525 & 1:28:33 \\ 
		3 & 0.068684 & 0.071334 & 0.071334 & 1:30:09 \\ 
		4 & 0.063998 & 0.065802 & 0.065802 & 1:26:00 \\ 
		5 & 0.060919 & 0.063771 & 0.063771 & 1:27:03 \\ 
		6 & 0.060293 & 0.062978 & 0.062978 & 1:26:51 \\ 
		\bottomrule
	\end{tabular}
	\caption{Pérdida de entrenamiento y validación para la reconstrucción de Xception}
	\label{tabla:resultados}
\end{table}

\begin{figure}[H]
	\centering
	\includegraphics[width=0.7\linewidth]{imagenes/curva_xception.png}
	\caption{Curva de aprendizaje con Xception}
\end{figure}

Observamos una más rápida convergencia que \textbf{ResNet34} motivado posiblemente por los filtros de distintos tamaños. Tenemos el mismo comportamiento de bondad entre la pérdida de entrenamiento y validación. Sin embargo, obtenemos una pérdida en validación de $E_{val} \approx 0.062$ la cual es superior a la de \textbf{ResNet34} indicando que a pesar de la rápida convergencia el resultado final es mejor con un mayor número de conexiones residuales. 

A continuación observamos su reconstrucción con un resultado muy similar en apariencia.
\begin{figure}[H]
	\centering
	\includegraphics[width=0.4\linewidth]{imagenes/reconstruccion_xception.png}
	\caption{Reconstrucción de las imágenes con Xception}
\end{figure}

A continuación, ante los resultados de esta comparación mostramos la siguiente tabla.

\begin{table}[H]
	\centering
	\begin{tabular}{|cccccc|}
		\toprule
		Arquitectura & $E_{train}$ & $E_{val}$ \\ 
		\midrule
		\textbf{ResNet34} & \textbf{0.055986} & \textbf{0.058405} \\ 
		Xception & 0.060293 & 0.062978 \\ 
		\bottomrule
	\end{tabular}
	\caption{Pérdida de entrenamiento y validación para clasificación con la parte convolucional congelada}
	\label{tabla:resultados3}
\end{table}

Eligiendo por tanto a \textbf{ResNet34} como codificador de las arquitecturas. 

\section{Clasificación}

\subsection{Entrenamiento en clasificación}

A continuación, mostramos los experimentos realizados para la tarea de clasificación. Tomamos \textbf{ResNet34} como codificador le añadimos la representación latente y las capas densamente conectadas, entrenamos las capas fully-connected congelando las capas convolucionales (codificador y representación latente) durante $3$ épocas tomando $\approx 3$ horas con estos resultados.

\begin{table}[H]
	\centering
	\begin{tabular}{|cccccc|}
		\toprule
		epoch & train\_loss & valid\_loss & accuracy & balanced\_accuracy & time \\ 
		\midrule
		0 & 0.112794 & 0.178968 & 0.805254 & 0.781898 & 1:04:50 \\ 
		1 & 0.119699 & 0.174729 & 0.819330 & 0.773066 & 1:03:01 \\ 
		\textbf{2} & \textbf{0.083459} & \textbf{0.141255} & \textbf{0.855540} & \textbf{0.794527} & \textbf{1:04:18} \\ 
		\bottomrule
	\end{tabular}
	\caption{Pérdida de entrenamiento y validación para clasificación con la parte convolucional congelada}
	\label{tabla:resultados3}
\end{table}

\begin{figure}[H]
	\centering
	\includegraphics[width=0.7\linewidth]{imagenes/task1_freeze.png}
	\caption{Curva de aprendizaje para clasificación parte convolucional congelada}
\end{figure}

Observamos una rápida convergencia al inicio del entrenamiento, estabilizándose durante la época número $1$ y volviendo a converger ligeramente durante la época $2$. En este caso y para esta primera fase, el entrenamiento no es tan óptimo, habiendo cierta distancia entre las pérdidas de validación y entrenamiento. 

A continuación, tras estas $3$ épocas ajustando las capas densamente conectadas pasamos a descongelar toda la red para que quede mejor ajustado.

\begin{table}[H]
	\centering
	\begin{tabular}{|cccccc|}
		\toprule
		epoch & train\_loss & valid\_loss & accuracy & balanced\_accuracy & time \\
		\midrule
		0 & 0.056434 & 0.209338 & 0.855477 & 0.840139 & 1:01:57 \\ 
		\textbf{1} & \textbf{0.040761} & \textbf{0.190361} & \textbf{0.864976} & \textbf{0.812309} & \textbf{1:01:41} \\ 
		2 & 0.023256 & 0.260177 & 0.876575 & 0.832247 & 1:00:26 \\ 
		3 & 0.009723 & 0.330192 & 0.877014 & 0.834721 & 1:00:30 \\ 
		\bottomrule
	\end{tabular}
	\caption{Pérdida de entrenamiento y validación para clasificación toda la red descongelada}
	\label{tabla:resultados3}
\end{table}

\begin{figure}[H]
	\centering
	\includegraphics[width=0.7\linewidth]{imagenes/task1_unfreeze.png}
	\caption{Curva de aprendizaje para clasificación toda la red descongelada tras ajustar capas densamente conectadas}
\end{figure}

Observamos como en las dos primeras épocas tenemos convergencia en validación, tras ellos se dispara validación indicándonos que hemos llegado al resultado tope en el entrenamiento. Este proceso de entrenamiento esta configurado con \textbf{EarlyStopping} por lo que PyTorch automáticamente guarda el mejor modelo (el que haya conseguido una pérdida en validación más baja en todas las épocas) en este caso tras descongelar obtiene como modelo final el correspondiente a la época número $1$. 

Observamos como tras haber pasado el proceso de ajuste inicial de las capas densamente conectadas al descongelar todo, obtenemos unas pérdida de validación y entrenamiento mucho más dispares. Esto podría cómo las diferentes partes de la arquitectura inducen complejidad en el proceso de entrenamiento: las capas densamente conectadas (que era lo único entrenable en la primera fase) hacen que esta diferencia sea muy pequeña porque estas capas son en comparación pequeñas. Al descongelar toda las capas de la red, siendo el codificador mucho mayor que las capas densas se observa como se induce un ruido mayor.


A continuación, y a pesar de que la teoría indica que primero ajustemos las capas densamente conectadas primero congelando el resto de las capas en sus inicios, probamos a entrenar toda la red descongelada sin un ajuste previo.

\begin{table}[H]
	\centering
	\begin{tabular}{|cccccc|}
		\toprule
		epoch & train\_loss & valid\_loss & accuracy & balanced\_accuracy & time \\
		\midrule
		0 & 0.080457 & 0.178836 & 0.854348 & 0.796523 & 1:23:49 \\ 
		1 & 0.084583 & 0.175143 & 0.847827 & 0.781100 & 1:24:09 \\ 
		\textbf{2} & \textbf{0.086810} & \textbf{0.159217} & \textbf{0.852185} & \textbf{0.800501} & \textbf{1:26:20} \\ 
		3 & 0.081304 & 0.370888 & 0.832309 & 0.747997 & 1:21:31 \\ 
		\bottomrule
	\end{tabular}
	\caption{Pérdida de entrenamiento y validación para clasificación para todo el entrenamiento toda la red descongelada}
	\label{tabla:resultados6}
\end{table}


\begin{figure}[H]
	\centering
	\includegraphics[width=0.7\linewidth]{imagenes/task1_totally_unfreeze.png}
	\caption{Curva de aprendizaje para clasificación todo el entrenamiento toda la red descongelada}
\end{figure}

Observamos como el mejor resultado que obtenemos es en la época número $2$ siendo peor que la mejor con la estrategia anterior. Adicionalmente, observamos como se alcanza una convergencia rápida llevando a fuerte overfitting justo en la siguiente época. Podemos interpretar como esta estrategia no sólo da peores resultados sino tiene una peor estabilidad en el proceso de entrenamiento.

\subsection{Validación en clasificación}

Tras entrenar los modelos pasamos a obtener los resultados finales infiriendo con el modelo en clasificación para los conjuntos de validación y test.

\subsubsection{Antes de aplicar votación}

En primer lugar calculamos la bondad sin aplicar el esquema de votación, la esperanza que se tiene es que tras la aplicación el esquema de votación los resultados sean iguales o mejores que los siguiente. No aplicaremos test ya que la tarea real es clasificación binaria con el esquema por lo que no es relevante el resultado de test en estas condiciones. 

A continuación, vemos los resultados obtenidos de ajuste para las métricas en entrenamiento y validación sin votación. Construimos las siguientes matrices de confusión tras la inferencia de cada conjunto.
 
\begin{figure}[H]
	\centering
	\includegraphics[width=0.8\linewidth]{imagenes/task1_results_train.png}
	\caption{Matriz de confusión de entrenamiento sin votación}
\end{figure}

\begin{figure}[H]
	\centering
	\includegraphics[width=0.8\linewidth]{imagenes/task1_results_validation.png}
	\caption{Matriz de confusión de validación sin votación}
\end{figure}

\begin{table}[H]
	\centering
	\begin{tabular}{|ccc|}
		\toprule
		Conjunto & accuracy ($\%$) & balanced\_accuracy ($\%$) \\
		\midrule
		Entrenamiento & 94.1586 & 92.6266 \\ 
		Validación & 86.4976 & 81.2309 \\ 
		\bottomrule
	\end{tabular}
	\caption{Resultados de validación y entrenamiento sin votación}
	\label{tabla:resultados10}
\end{table}

Observamos un ajuste alto en los datos de entrenamiento indicando un buen proceso de entrenamiento y resultados competitivos pero mejorables en validación. Observamos como la clase más conflictiva es \textbf{No Tumor} con las otras dos clases de tumores, indicando que la tarea más difícil es la detección del tumor y no la caracterización de este.

\subsubsection{Tras aplicar votación}

Aplicamos el esquema de votación, en primer lugar necesitamos explorar el parámetro $threshold$ que definimos en la metodología. El parámetro $threshold$ mide la tolerancia que tiene el modelo para predecir a meningioma, cuando existe un mayor número de predicciones por imagen a meningioma que $threshold$ el modelo predice toda la resonancia a meningioma, y en caso contrario a glioblastoma.

Tras hacer inferencia y ver qué cantidad promedio de predicciones de cada clase obteníamos para toda la resonancia de unos cuantos ejemplos, observamos como las predicciones a meningiomas eran muchos menores que para los gliomas. Esto está motivado por la propia naturaleza de la cantidad de imágenes de tumores de cada clase debido a que los meningiomas son más pequeños que los gliomas en promedio. 

Como definimos en la metodología necesitamos un esquema que lidie con este desbalanceo. Por ello, descartamos valores altos de $threshold$ y probamos dos valores pequeños: para $threhold = 5$ y $threshold = 3$.

A continuación mostramos las matrices de confusión asociadas a la aplicación de estos dos esquemas de votación.

\begin{figure}[H]
	\centering
	\includegraphics[width=0.8\linewidth]{imagenes/task1_val_5.png}
	\caption{Matriz de confusión de validación con votación: $Meningiomas < 5$}
\end{figure}

\begin{figure}[H]
	\centering
	\includegraphics[width=0.8\linewidth]{imagenes/task1_val_3.png}
	\caption{Matriz de confusión de validación con votación: $Meningiomas < 3$}
\end{figure}

\begin{table}[H]
	\centering
	\begin{tabular}{|ccc|}
		\toprule
		Threshold & accuracy ($\%$) & balanced\_accuracy ($\%$) \\
		\midrule
		5 & 88.2352 & 81.9857 \\ 
		3 & 86.8779 & 83.0363  \\ 
		\bottomrule
	\end{tabular}
	\caption{Resultados de validación con votación y de la búsqueda}
	\label{tabla:resultados11}
\end{table}

Observamos como obtenemos un mejor accuracy balanceado con $threshold = 3$, así que optamos por su elección.
Tras haber configurado $threshold = 3$ y haber visto su bondad en validación solo queda obtener el resultado final con test.

\begin{figure}[H]
	\centering
	\includegraphics[width=0.8\linewidth]{imagenes/task1_test.png}
	\caption{Matriz de confusión de Test con votación}
\end{figure}

Obtenemos resultados finales para test, obteniendo un accuracy del $84.6519 \%$ y un accuracy balanceado del $81.5556 \%$. Con este resultado sabemos que la obtención de un modelo final debería tener un accuracy balanceado en clasificación $ Acc_{F} \geq 81.5556 \% $.

\subsection{Comparativa de clasificación con el estado del arte}

Para finalizar la experimentación de esta primera tarea ponemos en contexto estos resultados con el estado del arte. Aunque no sean del todo comparables debido no son formulaciones del problema idénticas (se tiene otro tipo de tumor en el estado del arte) y se utilizan conjuntos de datos distintos T1-CE image en el estado del arte frente a BraTS en el este trabajo. Sin embargo, es importante hacer una observación de la bondad de los resultados teniendo en cuenta estas diferencias. 

\begin{table}[H]
	\centering
	\begin{tabular}{|cccc|}
		\toprule
		Trabajo & accuracy ($\%$) & balanced\_accuracy ($\%$) & $P_{test}$ \\
		\midrule
		\cite{cheng2015enhanced} & 91.2 & - & 61\\ 
		\cite{cheng2016retrieval} & 94.7 & - & 61\\ 
		\cite{abiwinanda2019brain} & 84.1 & - & 70\\ 
		\cite{pashaei2018brain} & 81.0 & - & 70\\ 
		\cite{sultan2019multi} & 96.1 & - & 75\\
		\cite{diaz2021deep} & 97.3 & - & 46 \\  
		Nuestro método & 84.65 & 81.56 & 632 \\ 
		\bottomrule
	\end{tabular}
	\caption{Resultados de validación con votación y de la búsqueda}
	\label{tabla:resultados12}
\end{table}

Los resultados del estado del arte no incluyen un validación con un esquema balanceado lo cual lo hace soluciones mucho menos generales y reales. Añadimos a la tabla, la entrada $P_{test}$ que la cantidad de pacientes que se usaron en el conjunto de test de cada trabajo.

Nuestro trabajo está validado en un conjunto de test $\approx 8$ veces mayor que el mayor conjunto utilizado en la literatura y utiliza una política de aprendizaje balanceado lo cual puede hacer obtener peores resultados en un métrica más débil como puede ser accuracy sin balanceo. Aún así nuestro método supera a 2 de los 6 métodos presentados en la revisión histórica.

\section{Segmentación}

\subsection{Entrenamiento en segmentación}

\begin{table}[H]
	\centering
	\begin{tabular}{|ccccc|}
		\toprule
		epoch & train\_loss & valid\_loss & Dice Loss & time \\ 
		\midrule
		0 & 0.095941 & 0.319656 & 0.319656 & 1:34:08 \\ 
		1 & 0.073242 & 0.258797 & 0.258797 & 1:32:56 \\ 
		2 & 0.059904 & 0.284256 & 0.284256 & 1:34:48 \\ 
		3 & 0.050076 & 0.274452 & 0.274452 & 1:37:00 \\ 
		\textbf{4} & \textbf{0.046783} & \textbf{0.266702} & \textbf{0.266702} & \textbf{1:40:18} \\ 
		\bottomrule
	\end{tabular}
	\caption{Pérdida de entrenamiento y validación para segmentación toda la red descongelada}
	\label{tabla:resultados4}
\end{table}

\begin{figure}[H]
	\centering
	\includegraphics[width=0.7\linewidth]{imagenes/curva_segmentation.png}
	\caption{Curva de aprendizaje para la tarea de segmentación}
\end{figure}

\begin{figure}[H]
	\centering
	\includegraphics[width=0.5\linewidth]{imagenes/output_segmentation.png}
	\caption{Comparación de la salida del modelo respecto la real}
\end{figure}

\subsection{Validación en segmentación}

\begin{table}[H]
	\centering
	\begin{tabular}{|cccc|}
		\toprule
		Partición & Similaridad Dice & Distancia Hausdorff & Sensibilidad \\ 
		\midrule
		Validación & 0.777343 & 14.573561 & 0.720415 \\ 
		Test & 0.772642 & 14.577154 & 0.709075 \\ 
		\bottomrule
	\end{tabular}
	\caption{Resultados de hold-out en validación y test para segmentación}
	\label{tabla:resultados5}
\end{table}

\begin{figure}[H]
	\centering
	\includegraphics[width=0.75\linewidth]{imagenes/dist_dice_val.png}
	\caption{Distribución de similaridad Dice en el conjunto de validación}
\end{figure}

\begin{figure}[H]
	\centering
	\includegraphics[width=0.75\linewidth]{imagenes/dist_haus_val.png}
	\caption{Distribución de distancia Hausdorff en el conjunto de validación}
\end{figure}

\begin{figure}[H]
	\centering
	\includegraphics[width=0.75\linewidth]{imagenes/dist_sen_val.png}
	\caption{Distribución de sensibilidad en el conjunto de validación}
\end{figure}

\begin{figure}[H]
	\centering
	\includegraphics[width=0.75\linewidth]{imagenes/dist_dice_test.png}
	\caption{Distribución de similaridad Dice en el conjunto de test}
\end{figure}

\begin{figure}[H]
	\centering
	\includegraphics[width=0.75\linewidth]{imagenes/dist_haus_test.png}
	\caption{Distribución de distancia Hausdorff en el conjunto de test}
\end{figure}

\begin{figure}[H]
	\centering
	\includegraphics[width=0.75\linewidth]{imagenes/dist_sen_test.png}
	\caption{Distribución de sensibilidad en el conjunto de test}
\end{figure}
\chapter{Conclusiones y Trabajos Futuros}

En este capítulo abordamos las conclusiones generales y específicas del trabajo. Adicionalmente, se expone qué mejoraríamos de este o cómo lo ampliaríamos en un trabajo futuro.

\section{Conclusiones del trabajo}

A continuación, comenzamos por describir las conclusiones del trabajo. Para ello, haremos tres apartados que ha sido clave para entender este trabajo: los resultados obtenidos en contexto con los recursos disponibles, el preprocesado aplicado (es decir, qué conclusión podemos extraer de la reducción de imágenes realizada para hacer el trabajo viable) y cómo la reconstrucción previa ha sido clave acelerando los tiempos de entrenamiento. 

\subsection{Resultados y recursos}

En este trabajo desde el primero momento hemos estado en desventaja ante otros autores por la falta de recursos y capacidades, ya que el único hardware personal que nos podría permitir entrenar era un PC personal. 

Sin embargo, un PC personal estaba descartado como posible hardware y la alternativa era usar el entorno de Kaggle que tiene sus limitaciones. 

Los resultados son moderadamente peores que la parte más reciente del estado del arte. No obstante, interpretamos que esto se debe a la falta de un entrenamiento más potente.  

\subsection{Nuevo preprocesado en el problema}

Para poder entrenar con los datos se tuvo que hacer un nuevo preprocesado que consistía en eliminar partes que podrían parecer redundantes. Se redujo las imágenes de entrenamiento a imágenes a aquellas con solo contenían lesión tumoral más una parte balanceada sin lesión. Esta reducción supone una nueva forma de preprocesar este conjunto de datos, y ante los resultados podemos considerar que se han mantenido estables, validando en la práctica este preprocesado.

Podemos concluir que la eliminación de la prueba \textbf{T2W} tampoco ha influido significativamente en los resultados. Pudiendo significar la existencia de información redundante en el conjunto de datos.

Por otro lado, tras los resultados no vemos unos fuertes efectos de la reducción en la dimensionalidad. La mayoría de los trabajos recientes usan 3D, usando toda la resonancia como entrada a la red. Descartamos esta opción por recursos, pero en la práctica no vemos una diferencia sustancial que haga que la información espacial entre imágenes de una resonancia permita segmentar mejor.


\subsection{El poder de la reconstrucción previa}

Finalmente, uno de los aspectos clave para hacer posible este trabajo es gran aceleración que hemos obtenido en el entrenamiento mediante la reconstrucción de las imágenes previa para inicializar la red.

El preentrenamiento con el autoencoder ha permitido que el codificador y representación latente capten características relevantes y patrones inherentes a los datos antes de que se entrene para la tarea de clasificación o segmentación. Esto es especialmente útil cuando se dispone de una cantidad limitada de datos etiquetados como es en el caso de datos médicos. Esta técnica consiste en aprovechar al máximo las imágenes de entrada.

En los experimentos hemos visto una gran convergencia en entrenamiento motivado por este paso previo clave en el proceso.

\section{Trabajos futuros}

A continuación, detallamos los aspectos en los que este trabajo podría ampliarse o mejorar. En primer lugar, hacemos un listado de aspectos específicos de mejora.

\begin{enumerate}
	\item \textbf{Mejora del hardware}. En un futuro necesitaríamos un hardware propio y suficiente con el podamos volver a entrenar el modelo más tiempo.
	\item \textbf{Mejorar la exploración de modelos e hiperparámetros}. Consecuencia de la mejora anterior permitiría una investigación más guiada experimentalmente.
	\item \textbf{Seguir investigando formas de abordar la tercera tarea}. La falta de datos y recursos impidieron poder resolver la predicción de la evolución del tumor. Una línea interesante es seguir investigando formas de aumento o creación de datos para llevarla a cabo.
\end{enumerate}

\subsection{Uso de transformers}

En el estado del arte hemos visto que el uso de transformer es una tendencia al alza que puede mejorar los resultados rompiendo la localidad de las convoluciones creando codificadores más potentes. Sería interesante probar su bondad en un futuro.  

\subsection{Unificación de arquitecturas}

En todo el trabajo hemos usado un codificador y representación latente común a las dos tareas y en la teoría también común a la tercera. Un trabajo a futuro sería investigar la bondad de ambos tres modelos para diferentes tareas optimizando esta parte común junto a los decoder o capas densamente conectadas de cada una en un mismo proceso de entrenamiento. Parte de la literatura ha indicado que esto bien ejecutado podría tener un efecto regularizador y reducir el ruido irreducible que veíamos en la optimización.

Por otro lado, esta puede ser una fuerte motivación a usar una arquitectura transformadora ya que en trabajos recientes han sido utilizadas como modelo multi-modal.

\subsection{Exploración de otras técnicas de aprendizaje no supervisado}

En la literatura se usan otras técnicas de aprendizaje no supervisado como las redes generativas adversarias, un trabajo futuro podría ser la investigación de la mejora de los modelos de este proyecto mediante ese tipo de aumento de datos.


\nocite{*}
\bibliographystyle{apalike}
\addcontentsline{toc}{chapter}{Bibliografía}
\bibliography{bibliografia/bibliografia}



\appendix

\chapter{Códigos y repositorio}

En este apéndice incluiremos las URL a los repositorios utilizados para el control de versiones del proyecto en GitHub y en Kaggle. Adicionalmente, incluiremos un vídeo de YouTube donde se prueba a la interfaz.

\section{Repositorio de GitHub}

En primer lugar, entramos a indicar cómo la estructura del repositorio de GitHub. En la raíz del repositorio tenemos tres carpetas:

\begin{itemize}
	\item \textbf{inference}. Definida como la carpeta dedicada a la inferencia con los modelos. De ella cuelga la estructura de carpetas de toda la interfaz. 
	\item \textbf{latex}. Carpeta dedicada a la memoria en latex.
	\item \textbf{train}. Carpeta dedicada a todo el proceso de entrenamiento y experimentación. En esta carpeta se encuentran los distintos notebooks que componen la construcción de los modelos y experimentos realizados. 
\end{itemize}

Puede consultar este repositorio de GitHub en la siguiente dirección \textbf{https://github.com/jucls/TFGBraTS} o haciendo click en el siguiente hipervínculo: \href{https://github.com/jucls/TFGBraTS}{Ir al código}.

\section{Notebooks en Kaggle}

Adicionalmente, aunque los notebooks de los experimentos son incluidos en el repositorio de GitHub también los haremos públicos en Kaggle. En la siguiente tabla incluimos los hipervínculos a cada uno.

Los links tienen la estructura común \textbf{https://www.kaggle.com/code/jaimecastillo}. Para buscar un notebook concreto solo es necesario añadirle la palabra de la columna Notebook de la siguiente tabla.

\begin{table}[H]
	\centering
	\begin{tabular}{|ccc|}
		\toprule
		Nombre & Notebook & Botón \\
		\midrule
		Preprocesado & preprocesado-encoder-y-tarea-1 & \href{https://www.kaggle.com/code/jaimecastillo/preprocesado-encoder-y-tarea-1}{Ir al código} \\
		Reconstrucción ResNet34 & building-encoder-with-resnet34 & \href{https://www.kaggle.com/jaimecastillo/building-encoder-with-resnet34}{Ir al código} \\
		Reconstrucción Xception & building-encoder-with-xception & \href{https://www.kaggle.com/jaimecastillo/building-encoder-with-xception}{Ir al código}\\   
		Clasificación & task-1-classification-with-resnet34 & \href{https://www.kaggle.com/jaimecastillo/task-1-classification-with-resnet34}{Ir al código}  \\ 
		Validación clasificación & task-1-test-postprocessing & \href{https://www.kaggle.com/jaimecastillo/task-1-test-postprocessing}{Ir al código} \\
		Segmentación & task-2-segmentation & \href{https://www.kaggle.com/jaimecastillo/task-2-segmentation}{Ir al código}  \\ 
		Validación segmentación & task-2-test-postprocessing & \href{https://www.kaggle.com/jaimecastillo/task-2-test-postprocessing}{Ir al código} \\ 
		\bottomrule
	\end{tabular}
	\caption{Enlaces a los notebooks en Kaggle}
	\label{tabla:notebooksKaggle}
\end{table}

\section{Modelos en Kaggle}

Los modelos obtenidos en este trabajo son muy pesados. No están en el repositorio de GitHub ya que tiene una capacidad limitada que estos sobrepasan. Puede descargar los checkpoints de los modelos obtenidos en este trabajo en Kaggle.

En la siguiente tabla recogemos los hipervínculos asociados a la descarga de cada checkpoint. La URL a buscar tiene esta estructura común a todos los modelos: \textbf{https://www.kaggle.com/datasets/jaimecastillo}. Para buscar un modelo solo es necesario añadirle la palabra de la columna Checkpoint de la siguiente tabla o hacer clic en el enlace.

\begin{table}[H]
	\centering
	\begin{tabular}{|ccc|}
		\toprule
		Notebook & Checkpoint & Link \\
		\midrule
		Reconstructor & encoder2 & \href{https://www.kaggle.com/datasets/jaimecastillo/encoder2}{Ir al modelo} \\
		Clasificador & task1model &  \href{https://www.kaggle.com/datasets/jaimecastillo/task1model}{Ir al modelo}\\
		Segmentador & segmentation & \href{https://www.kaggle.com/datasets/jaimecastillo/segmentation}{Ir al modelo}\\  
		\bottomrule
	\end{tabular}
	\caption{Enlaces a los checkpoints en Kaggle}
	\label{tabla:checkpointsKaggle}
\end{table}

\section{Demostración de uso de la interfaz}

A modo de demostración y tutorial del uso de la interfaz creamos un vídeo corto público en YouTube que puede ser consultado en el siguiente enlace:  ó haciendo click en el siguiente hipervínculo: \href{https://www.kaggle.com/datasets/jaimecastillo}{Ir al vídeo}.
\chapter{Documentación de la interfaz}


La interfaz tiene una finalidad simple y no es el objetivo del trabajo.  Sin embargo, en este apéndice detallamos el código de la interfaz: su estructura y las dependencias necesarias para poder ser utilizada.

\section{Estructura de la interfaz}

A continuación, comentamos la implementación y diseño de la interfaz. Mostraremos su diagrama de clases y comentaremos sus funciones.

\subsection{Diagrama de clases}

La interfaz tiene una estructura de ficheros modular separando un fichero principal \textbf{main} de otros ficheros que se encargan de la inferencia. Los programas que se encargan de utilizar los modelos para inferir son \textbf{model definition} y \textbf{predictions}. 
\begin{itemize}
	\item \textbf{Model definition}. Este fichero reúne el código necesario para definir los modelos.
	\item \textbf{Predictions}. Utiliza la instancia del \textbf{Learner} de \textbf{Model definition} para hacer la predicción final.
\end{itemize}

En un diagrama de clases mostraremos todos los ficheros representados como metaclases que contienen a su vez sus propias clases. Dentro de la definición de la metaclase o fichero \textbf{model definition} representaremos como un cuadro instancia todas las clases y métodos necesarios para las arquitectura y modelos que definimos en la Figura 4.1.

A continuación, mostraremos el diagrama de clases de toda la interfaz.

\begin{figure}[H]
	\centering
	\includegraphics[width=1.0\linewidth]{imagenes/diagrama_interfaz.png}
	\caption{Diagrama de clases de toda la interfaz}
\end{figure}

En la figura B.1, observamos el diagrama de clases de la interfaz. Las relaciones en este diagrama es como podemos ver la metaclase principal que instancia la ventana de la interfaz para interir utiliza a \textbf{predictions} y este a su vez utiliza a \textbf{model definition}. Además se observa como la metaclase de la definición del modelo utiliza a PyTorch y a FastAI. Y como la clase de la ventana principal usa \textbf{PyQt5} y \textbf{PyVista}. Finalmente, se observa como todas las clases usan a la librería \textbf{NiBabel}.


\subsection{Documentación de funciones}

En el siguiente apartado, creamos una lista con las funciones implicadas en cada clase. Comentaremos su función y lo que devuelve.

\begin{itemize}
	\item \textbf{predictions}
	\begin{itemize}
		\item \texttt{traerNII(ruta : String) : Tensor float32} \\
		Esta función recibe una cadena que representa la ruta de un archivo NII y devuelve un tensor de tipo float32.
		
		\item \texttt{clasify(ruta : String) : String} \\
		Esta función toma como parámetro una cadena con la ruta de un archivo y devuelve una cadena con la clasificación resultante.
		
		\item \texttt{segment(ruta : String) : NumPy Array} \\
		Esta función recibe una cadena que representa la ruta de un archivo y devuelve un arreglo de tipo NumPy con la segmentación resultante.
	\end{itemize}
	
	\item \textbf{main}
	\begin{itemize}
		\item \texttt{main()} \\
		Función principal que inicia la ejecución del programa.
	\end{itemize}
	
	\item \textbf{PointCloud}
	\begin{itemize}
		\item \texttt{PointCloud(nii\_data : NumPy Array)} \\
		Constructor de la clase \texttt{PointCloud} que recibe como parámetro un arreglo de tipo NumPy.
		
		\item \texttt{get\_points() : NumPy Array} \\
		Esta función devuelve un arreglo de tipo NumPy con los puntos de la nube de puntos.
		
		\item \texttt{get\_values() : NumPy Array} \\
		Esta función devuelve un arreglo de tipo NumPy con los valores correspondientes a la nube de puntos.
	\end{itemize}
	
	\item \textbf{model\_definition}
	\begin{itemize}
		\item \texttt{get\_learner\_classification() : Learner} \\
		Esta función devuelve un objeto de tipo \texttt{Learner} para la clasificación.
		
		\item \texttt{get\_learner\_segmentation() : Learner} \\
		Esta función devuelve un objeto de tipo \texttt{Learner} para la segmentación.
	\end{itemize}
	
	\item \textbf{MainWindow(QMainWindow)}
	\begin{itemize}
		\item \texttt{MainWindow()} \\
		Constructor de la clase \texttt{MainWindow}.
		
		\item \texttt{initUI()} \\
		Inicializa la interfaz de usuario.
		
		\item \texttt{create\_black\_container(self) : QtContainer} \\
		Crea un contenedor de color negro.
		
		\item \texttt{center()} \\
		Centra la ventana principal.
		
		\item \texttt{load\_file()} \\
		Carga un archivo.
		
		\item \texttt{update\_slices()} \\
		Actualiza las imágenes de las rebanadas.
		
		\item \texttt{display\_image(self, slice, label, index, type)} \\
		Muestra una imagen específica.
		
		\item \texttt{clasifica()} \\
		Clasifica los datos cargados.
		
		\item \texttt{segmenta()} \\
		Segmenta los datos cargados.
		
		\item \texttt{exporta()} \\
		Exporta los datos.
		
		\item \texttt{edita()} \\
		Edita los datos.
		
		\item \texttt{close\_application()} \\
		Cierra la aplicación.
	\end{itemize}
	
	\item \textbf{InfoWindow(QDialog)}
	\begin{itemize}
		\item \texttt{InfoWindow(info : String)} \\
		Constructor de la clase \texttt{InfoWindow} que recibe una cadena con información.
	\end{itemize}
\end{itemize}


\section{Dependencias}

La interfaz gráfica desarrollada para este proyecto está construida utilizando una serie de bibliotecas:

\begin{itemize}
	\item \textbf{PyTorch} y \textbf{FastAI}: Estas bibliotecas facilitan la inferencia de los modelos. Utilizamos a PyTorch para construir la definición del modelo, tras ello instanciamos el objeto Learner en FastAI al cual le cargamos los modelos.
	\item \textbf{PyVista}: Utilizada para la visualización de datos tridimensionales, PyVista ofrece una representación gráfica interactiva del cerebro como una nube de puntos, permitiendo una exploración detallada de las estructuras cerebrales. En la interfaz se utiliza para hacer la reconstrucción 3D a partir de las imágenes de la resonancia.
	
	\item \textbf{NiBabel}: Esta biblioteca es esencial para la manipulación y el procesamiento de imágenes médicas en formato NIfTI. Se utiliza para la lectura del formato \textbf{.nii} y el guardado también de las segmentaciones en este formato especializado.
	
	\item \textbf{PyQt5}: PyQt5 es la base sobre la cual se ha construido la interfaz gráfica de usuario (GUI). Dentro del desarrollo de interfaces gráficas es un estándar por su estética y facilidad de uso. Proporciona un marco robusto y flexible para el desarrollo de aplicaciones de escritorio interactivas en Python.
\end{itemize}

\chapter{Manual de uso de la interfaz}

\section{Clasificando}

\section{Segmentando}

\chapter*{}
\thispagestyle{empty}

\end{document}
