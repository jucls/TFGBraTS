\chapter{Códigos y repositorio}

En este apéndice incluiremos las URL a los repositorios utilizados para el control de versiones del proyecto en GitHub y en Kaggle. Adicionalmente, incluiremos un vídeo de YouTube donde se prueba a la interfaz.

\section{Repositorio de GitHub}

En primer lugar, entramos a indicar cómo la estructura del repositorio de GitHub. En la raíz del repositorio tenemos tres carpetas:

\begin{itemize}
	\item \textbf{inference}. Definida como la carpeta dedicada a la inferencia con los modelos. De ella cuelga la estructura de carpetas de toda la interfaz. 
	\item \textbf{latex}. Carpeta dedicada a la memoria en latex.
	\item \textbf{train}. Carpeta dedicada a todo el proceso de entrenamiento y experimentación. En esta carpeta se encuentran los distintos notebooks que componen la construcción de los modelos y experimentos realizados. 
\end{itemize}

Puede consultar este repositorio de GitHub en la siguiente dirección \textbf{https://github.com/jucls/TFGBraTS} o haciendo click en el siguiente hipervínculo: \href{https://github.com/jucls/TFGBraTS}{Ir al código}.

\section{Notebooks en Kaggle}

Adicionalmente, aunque los notebooks de los experimentos son incluidos en el repositorio de GitHub también los haremos públicos en Kaggle. En la siguiente tabla incluimos los hipervínculos a cada uno.

Los links tienen la estructura común \textbf{https://www.kaggle.com/code/jaimecastillo}. Para buscar un notebook concreto solo es necesario añadirle la palabra de la columna Notebook de la siguiente tabla.

\begin{table}[H]
	\centering
	\begin{tabular}{|ccc|}
		\toprule
		Nombre & Notebook & Botón \\
		\midrule
		Preprocesado & preprocesado-encoder-y-tarea-1 & \href{https://www.kaggle.com/code/jaimecastillo/preprocesado-encoder-y-tarea-1}{Ir al código} \\
		Reconstrucción ResNet34 & building-encoder-with-resnet34 & \href{https://www.kaggle.com/jaimecastillo/building-encoder-with-resnet34}{Ir al código} \\
		Reconstrucción Xception & building-encoder-with-xception & \href{https://www.kaggle.com/jaimecastillo/building-encoder-with-xception}{Ir al código}\\   
		Clasificación & task-1-classification-with-resnet34 & \href{https://www.kaggle.com/jaimecastillo/task-1-classification-with-resnet34}{Ir al código}  \\ 
		Validación clasificación & task-1-test-postprocessing & \href{https://www.kaggle.com/jaimecastillo/task-1-test-postprocessing}{Ir al código} \\
		Segmentación & task-2-segmentation & \href{https://www.kaggle.com/jaimecastillo/task-2-segmentation}{Ir al código}  \\ 
		Validación segmentación & task-2-test-postprocessing & \href{https://www.kaggle.com/jaimecastillo/task-2-test-postprocessing}{Ir al código} \\ 
		\bottomrule
	\end{tabular}
	\caption{Enlaces a los notebooks en Kaggle}
	\label{tabla:notebooksKaggle}
\end{table}

\section{Modelos en Kaggle}

Los modelos obtenidos en este trabajo son muy pesados. No están en el repositorio de GitHub ya que tiene una capacidad limitada que estos sobrepasan. Puede descargar los checkpoints de los modelos obtenidos en este trabajo en Kaggle.

En la siguiente tabla recogemos los hipervínculos asociados a la descarga de cada checkpoint. La URL a buscar tiene esta estructura común a todos los modelos: \textbf{https://www.kaggle.com/datasets/jaimecastillo}. Para buscar un modelo solo es necesario añadirle la palabra de la columna Checkpoint de la siguiente tabla o hacer clic en el enlace.

\begin{table}[H]
	\centering
	\begin{tabular}{|ccc|}
		\toprule
		Notebook & Checkpoint & Link \\
		\midrule
		Reconstructor & encoder2 & \href{https://www.kaggle.com/datasets/jaimecastillo/encoder2}{Ir al modelo} \\
		Clasificador & task1model &  \href{https://www.kaggle.com/datasets/jaimecastillo/task1model}{Ir al modelo}\\
		Segmentador & segmentation & \href{https://www.kaggle.com/datasets/jaimecastillo/segmentation}{Ir al modelo}\\  
		\bottomrule
	\end{tabular}
	\caption{Enlaces a los checkpoints en Kaggle}
	\label{tabla:checkpointsKaggle}
\end{table}

\section{Demostración de uso de la interfaz}

A modo de demostración y tutorial del uso de la interfaz creamos un vídeo corto público en YouTube que puede ser consultado en el siguiente enlace: \textbf{https://youtu.be/zO6oS9FLhyA?si=ewN1tjjCp1G1tjbs}  ó haciendo click en el siguiente hipervínculo: \href{https://youtu.be/zO6oS9FLhyA?si=ewN1tjjCp1G1tjbs}{Ir al vídeo}.