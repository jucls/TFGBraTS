\chapter{Metodología}

En este capítulo describimos en profundidad todos los pasos seguidos en los métodos empleados en el trabajo y su justificación. Posteriormente, se aplicarán en la experimentación.

\section{Análisis de los recursos disponibles}


\section{Preprocesado de Datos}

En este apartado se explicará el preprocesamiento que se ha aplicado a las resonancias magnéticas para convertirlas a entradas de los modelos. 


Partiendo de nuestro conjunto de datos que presentamos en la introducción obtenido de la competición BraTS en Synapse. Ya vemos como las resonancias presentan características favorables para ser una entrada a la red.

\begin{enumerate}
	\item \textbf{Dimensiones estandarizadas} : Todas las resonancias (adultos, niños, diferente tipo de tumor) presentan las mismas dimensiones.
	\item \textbf{Imágenes estandarizadas} : Dado todas las resonancias se han hecho con el mismo estándar de escáner, todas presentan el mismo rango para su visualización. 
	\item \textbf{No existen valores faltantes} : Observamos como el conjunto de datos es completo en su definición, todas las resonancias de cada paciente tienen las mismas cuatro pruebas.
\end{enumerate}

\subsection{Normalizado de las imágenes}

Las imágenes que componen las resonancias son mapas en escala de gris donde un píxel de la imagen puede tomar un valor de gris en el intervalo $[0, 256)$. Entre las imágenes de distintas resonancias se encuentra una misma distribución de valores de píxeles para representar la misma información. Sin embargo, el proceso de entrenamiento no deja de ser un proceso de optimización y puede que este rango sea aún demasiado grande.

Adicionalmente, para evitar posibles píxeles erróneos en la toma de las imágenes que podamos interpretar como outliers que tengan un impacto negativo en el entrenamiento y para hacer las imágenes más interpretables se aplica a las imágenes normalización Z-score o estandarización.

$$ X_{std}^{i}= \frac{x^{i}-mean}{std} $$



\subsection{Recortado de imagen}

Podría ser razonable reducir las dimensiones de las imágenes para hacer a nuestros datos menos pesados. Sin embargo, se opta por no hacerlo por seguridad y escalibilidad. BraTS fija esas dimensiones en base del estándar en una resonancia magnética, así para cualquier paciente se garantiza que la imagen de su cerebro se puede representar en una resonancia en unas condiciones de resolución iguales al resto de pacientes.

Si recortamos las imágenes de forma cuadrada al cerebro más grande de todas las resonancias, podríamos encontrarnos en inferencia con un cerebro mayor que no se podría representar en una imagen. Es necesario dejar cierto margen, optando por respetar el margen inicial que marcan los organizadores médicos de BraTS.

