\chapter{Introducción}

Los tumores cerebrales son una de las formas más letales de cáncer. Específicamente, los glioblastomas y sus variantes difusas son los más comunes y agresivos tipos de tumor del sistema nervioso central en adultos. Su alta heterogeneidad en apariencia, forma e histología los convierte en una de las patologías más difíciles de diagnosticar, de tratar y un reto para el campo de la imagen médica.

Desde el punto de vista de la ingeniería y la informática, vemos como sin duda la aplicación de técnicas de Visión por Computador es una de las máximas para la investigación en imagen médica en la actualidad. Sólo considerando su aplicación en el diagnóstico de enfermedades, desde 2008 el número de publicaciones promedio realizadas por año se ha incrementado notablemente tanto que actualmente es diez veces mayor que en sus inicios. 

Resultados notables como la inclusión de robots especializados para la cirugía \cite{cheng2022vinci} o buenos resultados en competiciones de ciencia de datos que replican la precisión médica en el diagnóstico mediante imagen \cite{bulten2022artificial} evidencian esta tendencia. El trabajo conjunto de personal médico e ingenieros promete seguir dando resultados que de forma separada eran inaccesibles.



\section{Objetivos}


Con este trabajo se persigue la creación de una arquitectura basada en aprendizaje profundo para equipar a un programa de uso médico, estudiarla y compararla junto a trabajos previos y estado del arte. Este programa tiene el objetivo de la ayuda en la evaluación del diagnóstico y pronóstico de un posible paciente de tumor cerebral y en caso afirmativo, la ayuda en la aplicación de la terapia por radiación. 

Se seguirá un planteamiento similar al seguido en la competición \textbf{BraTS: Brain Tumor Segmentation 2023} \cite{baid2021rsna} históricamente reconocida por ser un benchmark recurrente de las capacidades de las arquitecturas profundas en el campo de la imagen médica.

BraTS es una competición que se define como un conglomerado de diferentes tareas relevantes en el diagnóstico de los tumores cerebrales << Cluster of Challenges >>. Este año dando especialmente importancia a la generalidad de un modelo que mantenga los resultados anteriores para pacientes más diversos (de diferente origen étnico y de diferentes edades) y con diferentes tipos de tumores. En concreto para 2023, se contemplaron las siguientes tareas por separado: segmentación de glioblastomas en adultos, segmentación de meningiomas en adultos, segmentación de tumores pediátricos, segmentación de tumores de pacientes de origen africano, generación de pruebas faltantes y generación de partes de la imagen.

De forma análoga a esta competición, se plantea conseguir dicho objetivo a partir de la resolución de las siguientes tareas.

\begin{enumerate}
	\item \textbf{Segmentación de los tumores.} 
	La segmentación de los glioblastomas y los meningiomas, indicando sus zonas médicas reconocidas.
	\item \textbf{Clasificación entre tipos de tumores.} Clasificación binaria entre glioblastomas y meningiomas. El programa indicará de qué tipo de tumor se trata una prueba dada.
	\item \textbf{Predicción de la evolución.} La segmentación a corto plazo de la más probable instancia futura a partir de la resonancia actual. Ya no solo se pretende dar una segmentación y clasificación que pudieran aportar valor en las decisiones médicas, sino por predecir una nueva segmentación a partir de la resonancia apoyándose en casos similares y en la segmentación de la metástasis de la resonancia actual.
\end{enumerate}

A continuación, detallaremos de una forma más profunda la naturaleza de este planteamiento.

Sólo en los EEUU para 2024 se esperan 25400 nuevos casos de tumor cerebral. La supervivencia de estos a los cinco años es del $33.8 \%$ de los pacientes. \cite{cancerorg}

El cerebro no tiene terminaciones nerviosas. Los pacientes no sienten dolor a causa de un tumor cerebral por sí mismo, lo cual hace que no exista una alerta sobre el paciente que lo motive a buscar ayuda médica en las primeras fases de la patología. Generalmente, acaban buscando ayuda médica por la aparición de otros indicios relacionados difíciles de distinguir de otras patologías agudas y de menor transcendencia como visión borrosa, pérdida del control, etc. Además, los glioblastomas son tumores de muy rápido crecimiento pueden llegar a estar en una fase avanzada desde su inicio en tan solo 2-3 meses.

Por estos motivos, es común llegar tarde. Tomando mucha importancia el diagnóstico temprano para su superación. Es en este punto donde se tiene el objetivo de evaluar las capacidades del aprendizaje profundo para la segmentación de tumores que en sus inicios podrían ser pasados desapercibidos por incluso el ojo médico y eventualmente la segmentación de las zonas de interés de la aparición de nuevos tumores.

En general, los tumores cerebrales son difíciles de tratar y son resistentes a terapias convencionales usadas en otros tipos de cánceres como la quimioterapia debido a los desafíos que presenta el cerebro para tolerar ciertos químicos, transportar medicamentos dentro de él y la alta importancia que tiene en este órgano la optimización del uso de tratamientos que puedan ser invasivos. En otras palabras, el uso de tratamientos basados en la extirpación o en la medicación pueden ser arriesgados. Por tanto, el tratamiento más común de estos está basado en la radioterapia.

A la hora de aplicar un tratamiento de radioterapia siempre se tiene el objetivo de ser lo menos invasivo posible. Para ello, el médico debe ser lo más preciso posible en introducir la segmentación correcta en la que se aplicarán los rayos. 

Uno de los objetivos específicos para la ayuda en el tratamiento se basaría en el uso del modelo para automatizar esta tarea ya que podría suponer ahorrar costes en errores humanos y en tiempo a veces escaso para el personal médico cuya tarea se reduciría a corregir dicha segmentación. En nuestro caso, integrándolo en un programa de uso médico. 

Por otro lado, de forma general lo podemos interpretar como un \textbf{sistema de apoyo a la decisión médica} ya que en ningún caso respecto el desarrollo actual de este problema se pretende sustituir la decisión final del personal médico. Aunque sí aprovechar las capacidades que puede ofrecer el aprendizaje profundo en un mejor diagnóstico a través de caracterizar mejor el tejido afectado, clasificándolo y caracterizando la evolución del tejido afectado.


\section{Metodología}

\subsection{Conjunto de datos}

Una de las \textbf{limitaciones frecuentes en el campo de la imagen médica} es la poca disponibilidad de datos. En general y también para nuestro problema, las dificultades que se presentan a la hora de construir un conjunto de datos médico grande son:

\begin{enumerate}
	\item \textbf{Poca densidad de pruebas médicas.} A pesar de que la densidad de casos es alta, es frecuente que la cantidad de pruebas que se realizan sea mucho más baja. Especialmente, para datos médicos es frecuente encontrarse con pocos datos en magnitud con la necesidad de variabilidad en la muestra que tienen las técnicas típicas de optimización.
	
	\item \textbf{La desvinculación de los pacientes de sus datos}. El tratamiento de un dato médico siempre supone la eliminación de cualquier identificativo que ponga en riesgo la privacidad de este. Suponiendo un trabajo adicional para el personal médico que no siempre se puede asumir.
	
	\item \textbf{No existe una fuerte centralización de datos}. Al igual que los avances en imagen médica, el interés por la construcción de una base de datos única que recoja los máximo datos posibles es también reciente haciendo que en la actualidad la mayoría de los datos estén distribuidos en muchos centros médicos diferentes con formatos diferentes.
	
\end{enumerate}

Partimos del conjunto de datos \textbf{BraTS} y serán los que únicamente utilizaremos para todo el trabajo ya que son los únicos que se pueden encontrar en la red pidiendo un acceso a ellos de una forma simple e incluso podríamos decir legal.

Hasta nuestro conocimiento, BraTS es el mayor conglomerado de resonancias magnéticas en la actualidad para los desafíos que planteamos en este trabajo. Otros datasets usados por la comunidad para este problema como el incluido en \textbf{Medical Segmentation Decathlon} descubrimos que es un subconjunto de \textbf{BraTS}.


\textbf{BraTS} está patrocinado y organizado por la ASNR (American Society of Neurology), MICCAI (Medical Image Computing and Computer Assisted Intervention Society), National Cancer Institute, la Universidad de Pensilvania e Intel entre otros. 

\textbf{Los datos de BraTS son heterogéneos} ya no sólo externamente con diferentes tipos tumores (glioblastomas y meningiomas) y de pacientes de orígenes distintos y de diferentes edades sino también internamente con lesiones más y menos avanzadas en el tiempo (low- and high-grade) y con datos de multitud de centros que han sido escaneados con distintos escáneres.

Definiremos como nuestro conjunto de datos \textbf{$X$} a un conjunto de resonancias magnéticas cerebrales completas. Este lo tenemos en formato \textbf{nii : Neuroimaging Informatics Technology Initiative (NIfTI)} el cual podremos hacer operar con el directamente con su versión comprimida en \textbf{.gz : GNU ZIP} gracias al uso de la librería \textbf{nibabel} la cual nos permitirá la lectura y conversión de cada resonancia a un 3D-array de NumPy de dimensiones $240 \times 240 \times 155$ que representa un array de $155$ imágenes de resolución $240 \times 240$ del cerebro del paciente divida en partes equidistantes. 

Por otro lado, definimos el conjunto de etiquetas $Y$ como otro 3D-array de las mismas dimensiones que en el conjunto de datos donde se muestra la segmentación (Grout Truth) de los tumores.

En términos numéricos, contando solo los datos que tenemos sobre adultos tenemos $1251$ resonancias de glioblastomas y $1000$ resonancias de meningiomas. Adicionalmente, tenemos glioblastomas pediátricos

\subsubsection{Visualizado de datos}

A continuación, detallaremos más profundamente los datos haciendo algunas visualizaciones. 

\subsection{Línea de investigación}

** Explicar profundamente cuando lo tenga claro **

Unificación de las 3 tareas en un transformer. Uso de GAN para aumento de datos. Encoder-Decoder. Uso frecuente de U-nets.

\subsection{Evaluación y métricas}



