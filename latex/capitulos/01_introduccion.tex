\chapter{Introducción}

Los tumores cerebrales son una de las formas más letales de cáncer. Específicamente, los glioblastomas y sus variantes difusas son los más comunes y agresivos tipos de tumor del sistema nervioso central en adultos. Su alta heterogeneidad en apariencia, forma e histología los convierte en una de las patologías más difíciles de diagnosticar, de tratar y un reto para el campo de la imagen médica.

Desde el punto de vista de la ingeniería y la informática, vemos como sin duda la aplicación de técnicas de Visión por Computador es una de las máximas para la investigación en imagen médica en la actualidad. Sólo considerando su aplicación en el diagnóstico de enfermedades, desde 2008 el número de publicaciones promedio realizadas por año se ha incrementado notablemente tanto que actualmente es diez veces mayor que en sus inicios. 

Resultados notables como la inclusión de robots especializados para la cirugía \cite{DAVINCI} o buenos resultados en competiciones de ciencia de datos que replican la precisión médica \cite{PANDA} evidencian esta tendencia. El trabajo conjunto de personal médico e ingenieros promete seguir dando resultados que de forma separada eran inaccesibles.



\section{Objetivos}

**Hablar sobre porqué al dejar la tarea de segmentar a una máquina esta puede ser más precisa que un humano**

Con este trabajo se pretende perseguir la creación de una arquitectura que mejore el estado del arte actual para equipar a un programa al servicio de personal médico para la ayuda en la evaluación del diagnóstico de un posible paciente de tumor cerebral.

A continuación, detallaremos de una forma más profunda estos objetivos.

El cerebro no tiene terminaciones nerviosas. Los pacientes no sienten dolor a causa de un tumor cerebral por sí mismo. Generalmente, acaban buscando ayuda médica por la aparición de otros indicios relacionados difíciles de distinguir de otras patologías agudas y de mucho menor transcendencia (visión borrosa, pérdida del control, etc).

Los glioblastomas y sus variantes tienen una media de supervivencia de 15 meses tras su diagnóstico.

En general, los tumores cerebrales son difíciles de tratar y son resistentes a terapias convencionales usadas en otros tipos de cánceres como la quimioterapia debido a los desafíos que presenta el cerebro para tolerar ciertos químicos, transportar medicamentos dentro de él y la alta importancia que tiene en este órgano la optimización del uso de tratamientos que puedan ser invasivos. En otras palabras, el uso de tratamientos basados en la extirpación o en la medicación pueden ser arriesgados. Por tanto, el tratamiento más común de estos está basado en la radioterapia.


\section{Metodología}
