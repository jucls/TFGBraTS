\chapter*{}

\cleardoublepage
\thispagestyle{empty}


\begin{center}
	\large\bfseries \myTitle: \mySubTitle \\
\end{center}
\begin{center}
	\myName\\
\end{center}

%\vspace{0.7cm}
\noindent{\textbf{Palabras clave}: clasificación, segmentación, red convolucional, U-net, conexiones residuales, resonancia, slice, tumor, decoder, encoder, representación latente, aprendizaje no supervisado, transformer y autoencoder. }\\

\vspace{0.7cm}
\noindent{\textbf{Resumen}}\\

Los tumores cerebrales son uno de los cánceres más letales y generalmente desafiantes a la hora de diagnosticar. En este proyecto exploraremos las capacidades que ofrecen las técnicas de aprendizaje profundo sobre imágenes de resonancia magnética dando como resultado la implementación de arquitecturas para la ayuda en su diagnóstico y una interfaz a modo de demostración de uso médico.

El desafío es obtener una interpretación más profunda de un órgano tan complejo como es el cerebro para sus formas de cáncer. El diagnóstico de un tumor cerebral tiene distintas tareas asociadas: la \textbf{segmentación} de estos tumores tiene los objetivos de agilizar el proceso de segmentación manual al personal médico, ser un elemento de seguridad en este proceso y eventualmente poder identificar especialmente los que pueden pasar desapercibidos incluso por un especialista humano. La \textbf{clasificación} entre los dos principales tipos de tumores, \textbf{glioblastomas} (más agresivos) y \textbf{meningiomas} (menos agresivos), puede guiar al personal médico en los tumores más difíciles de distinguir a tomar decisiones en base al tratamiento a seguir. De forma teórica y siguiendo la línea del trabajo se propone la solución a una tercera tarea: la \textbf{predicción de la evolución} de los tumores en una instancia a corto plazo de tiempo. 

Estas tareas que conforman el diagnóstico de los tumores cerebrales se incluirán en este trabajo y se pondrán en valor junto con el estado del arte.

\cleardoublepage
\thispagestyle{empty}

\begin{center}
	\large\bfseries \myTitleENGLISH: \mySubTitleENGLISH \\
\end{center}
\begin{center}
	\myName \\
\end{center}

%\vspace{0.7cm}
\noindent{\textbf{Keywords}: classification, segmentation, convolutional network, U-net, residual connections, MRI, slice, tumor, decoder, encoder, latent representation, unsupervised learning, transformer and autoencoder.}\\

\vspace{0.7cm}
\noindent{\textbf{Abstract}}\\

Brain tumors are among the most lethal and generally challenging cancers to diagnose. In this project, we will explore the capabilities offered by deep learning techniques on magnetic resonance images, resulting in the implementation of architectures to assist in diagnosis and a demonstration interface for medical use.

The challenge is to obtain a deeper understanding of an organ as complex as the brain for its cancer forms. The diagnosis of a brain tumor involves various tasks: the \textbf{segmentation} of these tumors aims to expedite the manual segmentation process for medical staff, provide a safety element in this process, and eventually identify those that may go unnoticed even by a human specialist. The \textbf{classification} between the two main types of tumors, \textbf{glioblastomas} (more aggressive) and \textbf{meningiomas} (less aggressive), can guide medical staff in the more difficult-to-distinguish tumors to make decisions based on the treatment to follow. Theoretically, and in line with the work, a solution is proposed for a third task: the \textbf{prediction of the evolution} of the tumors in a short-term time frame.

These tasks that comprise the diagnosis of brain tumors will be included in this work and will be highlighted along with the state of the art.


\chapter*{}
\thispagestyle{empty}


\noindent\rule[-1ex]{\textwidth}{2pt}\\[4.5ex]

Yo, \textbf{\myName}, alumno de la titulación \myDegree de la \textbf{Escuela Técnica Superior
de Ingenierías Informática y de Telecomunicación de la Universidad de Granada}, con DNI 45924736S, autorizo la
ubicación de la siguiente copia de mi Trabajo Fin de Grado en la biblioteca del centro para que pueda ser
consultada por las personas que lo deseen.

\vspace{6cm}

\noindent Fdo: \myName

\vspace{2cm}

\begin{flushright}
Granada a 24 de Junio de 2024.
\end{flushright}


\chapter*{}
\thispagestyle{empty}

\noindent\rule[-1ex]{\textwidth}{2pt}\\[4.5ex]

Dra. \textbf{\myProf}, Profesora del Departamento de \myDepartment de la Universidad de Granada.


\vspace{0.5cm}

\textbf{Informan:}

\vspace{0.5cm}

Que el presente trabajo, titulado \textit{\textbf{\myTitle, \mySubTitle}},
ha sido realizado bajo su supervisión por \textbf{\myName}, y autorizamos la defensa de dicho trabajo ante el tribunal que corresponda.

\vspace{0.5cm}

Y para que conste, expiden y firman el presente informe en Granada a 25 de Junio de 2024.

\vspace{1cm}

\textbf{La directora:}

\vspace{5cm}

\noindent \textbf{}

\chapter*{Agradecimientos}
\thispagestyle{empty}
\vspace{1cm}


Desde el inicio de mi etapa universitaria y en este tan corto período de tiempo, puedo decir que los cambios en mí, en magnitud, han sido positivamente radicales. En este trabajo que representa el final de esta primera etapa, sólo puedo estar profundamente agradecido por todas las personas que se han cruzado de forma positiva en algún punto conmigo. A mi familia, que siempre ha creído en mí, dándome la oportunidad de poder formarme incluso cuando más me ha costado afrontar este proceso de cambio; a los buenos amigos que he hecho en el camino y a todos los buenos profesores de los que he recibido clase o apoyo, que han sido inspiradores para mí.
